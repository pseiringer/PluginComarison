\chapter*{Kurzfassung}

Das Arbeiten mit integrierten Entwicklungsumgebungen (IDEs)
kann durch die Entwicklung von 
Plugins oder Extensions verbessert werden. So ermöglichen sie unter
anderem die Automatisierung von Abläufen während der Entwicklung oder
die Einbindung von zusätzlichen Werkzeugen. Die Entscheidung,
für welche Entwicklungsumgebung ein Plugin entwickelt werden soll,
ist allerdings nicht trivial. Für diese Entscheidung ist es von Vorteil,
ein Basiswissen über den Aufbau und die Funktionalität der APIs
sowie über die Entwicklung solcher Plugins und Extensions zu haben.

In dieser Arbeit werden daher die Extension APIs der Entwicklungsumgebungen
Visual Studio Code
und IntelliJ IDEA analysiert und verglichen. Eingegangen wird
auf den Aufbau und die Funktionalität der APIs sowie 
auf die Testbarkeit und das Publishing von Plugins.
Für den Vergleich wird ein Katalog an Bewertungskriterien aufgestellt, 
mit denen die beiden Systeme beurteilt und verglichen werden können.

Um für den Vergleich eine praktische Grundlage zu schaffen, wird
ein prototypisches Plugin für beide Plattformen entwickelt. Aufgrund
der dabei gemachten Erfahrungen können die Bewertungskriterien
ausgeführt und verglichen werden. Durch das erarbeitete Wissen
kann eine Entscheidungsbasis für neue Plugin-EntwicklerInnen
geschaffen werden und der Einstieg in die Plugin-Entwicklung 
erleichtert werden.