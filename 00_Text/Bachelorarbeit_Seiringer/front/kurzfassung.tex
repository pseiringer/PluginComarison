\chapter{Kurzfassung}

Das Arbeiten mit IDEs kann durch die Entwicklung von 
Plugins oder Extensions verbessert werden. So ermöglichen sie unter
anderem die automatisierung von Abläufen während der Entwicklung oder
die Einbindung von zusätzlichen Werkzeugen in die bereits bestehende
Entwicklungsumgebung. Die Entscheidung,
für welche Programmierumgebung ein Plugin entwickelt werden soll,
ist allerdings nicht trivial. Für diese Entscheidung ist es von Vorteil,
ein Basiswissen über den Aufbau und die Funktionalität der APIs,
sowie über die Entwicklung solcher Plugins und Extensions zu haben.

Es werden daher die Extension-APIs in den IDEs Visual Studio Code
und IntelliJ IDEA analysiert und verglichen. Eingegangen wird
auf den Aufbau und die Funktionalität der APIs, sowie 
auf die Testbarkeit und das Publishing von Plugins.
Für den Vergleich wird ein Katalog an Bewertungskriterien aufgestellt, 
an denen die beiden Systeme gemessen werden können.

Um für den Vergleich eine praktische Grundlage zu schaffen, wird
ein prototypisches Plugin für beide Plattformen entwickelt. Aufgrund
der dabei gemachten Erfahrungen können die Bewertungskritierien
ausgeführt und verglichen werden. Durch das erarbeitete Wissen
kann effektiv eine Entscheidungsbasis für neue Plugin EntwicklerInnen
geschaffen werden und der Einstieg in die Pluginentwicklung 
erleichtert werden.