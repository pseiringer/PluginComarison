\chapter{Kurzfassung}

Das Arbeiten mit IDEs kann durch die Entwicklung von 
Plugins oder Extensions verbessert werden. Die Entscheidung,
für welche Programmierumgebung ein Plugin entwickelt werden soll,
ist, vor allem ohne Vorerfahrung, allerdings nicht trivial.

Es werden daher die Extension-APIs in den IDEs Visual Studio Code
und IntelliJ IDEA analysiert und verglichen. Für den Vergleich
wird ein Katalog an Bewertungskriterien aufgestellt, an denen die
beiden Systeme gemessen werden können.

Um für den Vergleich eine praktische Grundlage zu schaffen, wird
ein prototypisches Plugin für beide Plattformen entwickelt. Aufgrund
der dabei gemachten Erfahrungen können die Bewertungskritierien
ausgeführt und verglichen werden. Durch das erarbeitete Wissen
kann effektiv eine Entscheidungsbasis für neue Plugin EntwicklerInnen
geschaffen werden.