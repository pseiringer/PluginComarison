\chapter{Abstract}


\begin{english} %switch to English language rules

Working with IDEs can be vastly improved by plugins
or extensions. For example they allow for automation 
of development tasks, or for integration of additional development
tools into the existing development environment.
However, the decision for which IDE
a plugin should be developed, is (especially without any
previous experience) far from trivial. For this decision,
it is advantageous to have a fundamental knowledge of
the functionality of the APIs, as well as of how such
plugins and extensions are developed.

Therefore, the extension APIs in Visual Studio Code and 
IntelliJ IDEA will be analysed and compared. Especially
the structure and functionality of the APIs, but also
their testability and publishing methods will be looked into.
In order to make a comparison, a catalog of criteria will be created,
against which the two systems can be measured.

In order to create a practical basis for the comparison,
a prototypical plugin will be developed for both platforms.
Based on the experiences made while developing this 
prototype, the criteria can be evaluated and compared.
With the help of the aquired knowledge, an effective
basis for decision-making can be created for
new plugin developers and the entry into plugin development
can be facilitated.

\end{english}
