\chapter{Bewertungskriterien}
\label{cha:Kriterien}

\section{Popularität der Entwicklungsumgebung}
\label{sec:Kriterien_Popularität}

Die Popularität von VS Code und IntelliJ kann aufgrund
von verschiedenen Umfragen und beliebtheits-Indizes 
analysiert werden. Dabei darf allerdings nicht außer
Acht gelassen werden, dass es sich bei VS Code und
IntelliJ nicht zwingend um Konkurrenzprodukte handelt,
sondern die beiden IDEs unterschiedliche Zielgruppen ansprechen.
Natürlich ist auch zu beachten wie viele Plugins 
für die Entwicklungsumgebung existieren, und somit wie 
populär die Entwicklungsumgebung bei Plugin EntwicklerInnen ist. 


\section{Performance}
\label{sec:Kriterien_Performance}

Da es sich bei den IDEs um komplexe Systeme handelt und die 
entwickelten Plugins sehr eng mit diesen Systemen verwoben sind,
ist es nur schwer möglich die Performance der entwickelten
Plugins zu vergleichen. Allerdings kann die allgemeine
Performance der IDEs verglichen werden.


\section{Feature Umfang}
\label{sec:Kriterien_FeatureUmfang}

Hier soll verglichen werden, wie viele Funktionen
den EntwicklerInnen von den Plugin APIs 
bereits zur Verfügung gestellt werden.


\section{Intuitivität der API}
\label{sec:Kriterien_Intuitivität}

Die Intuitivität der Plugin APIs kann bewertet werden,
in dem verglichen wird, wie einfach es ist die
angebotenen Features zu verwenden.
Dabei muss unter anderem beachtet werden wie viel Vorwissen über die
API man mitbringen muss, wie viele Abhängigkeiten es
gibt die nicht von Anfang an klar sind und ob es 
häufige Fallstricke gibt, die EntwicklerInnen vermeiden müssen.


\section{Dokumentation der API}
\label{sec:Kriterien_Dokumentation}

Um eine Plugin API zu erlernen dient als erste Grundlage
immer die Dokumentation. Hier muss vor allem
auf Übersichtlichkeit und Vollstädingkeit großer Wert
gelegt werden.


\section{Testbarkeit des Plugins}
\label{sec:Kriterien_Testbarkeit}

Automatisierte Softwaretests zählen zu den wichtigsten
Standards in der Softwareentwicklung.
Um sicherzustellen, dass NutzerInnen eine gute User Experience
geboten wird, müssen die Plugin APIs das Testen der
Plugins nicht nur ermöglichen sondern möglichst
gut unterstützen.


\section{Möglichkeiten des Publishings}
\label{sec:Kriterien_Publishing}

Hier soll vorallem betrachtet werden, wie
einfach es für EntwicklerInnen ist ein Plugin zu 
veröffentlichen, aber auch welche Möglichkeiten es gibt 
das Plugin zu Vermarkten. Dabei ist nicht nur die Möglichkeit
auf Monetarisierung relevant, sondern auch die
Optionen für die Darstellung des Plugins im Marketplace 
(zum Beispiel die Einbindung von Logos oder Beschreibungstexten)


\section{Installationsprozess des Plugins}
\label{sec:Kriterien_Installationsprozess}

Zuletzt ist auch wichtig wie leicht es für potentielle 
NutzerInnen ist, das entwickelte Plugin zu finden und
in ihrer Programmierumgebung zu installieren.