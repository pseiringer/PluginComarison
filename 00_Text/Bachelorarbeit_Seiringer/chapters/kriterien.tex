\chapter{Bewertungskriterien}
\label{cha:Kriterien}

Die folgenden Bewertungskriterien decken einige Bereiche ab,
die, meiner Meinung nach, für die Arbeit mit den Entwicklungsumgebungen
und deren Plugin-APIs relevant sind. Dabei geht es nicht um eine 
konkrete Bewertung nach Punkten oder einem festen Schema, sondern mehr
um eine direkte Gegenüberstellung und den Vergleich, welche spezifischen
Punkte in den beiden Umgebungen besser oder schlechter 
umgesetzt sind. Die Einschätzung dieses Vergleichs erfolgt aufgrund
der durch die Entwicklung des Prototyps gesammelten Erfahrung.
% //TODO maybe: beschreiben warum allgemeine Bewertung nicht sinnvoll (vom projekt abhängig usw...)

\section{Popularität der Entwicklungsumgebung} 
\label{sec:Kriterien_Popularität}    
Die Popularität von VS Code und IntelliJ kann aufgrund
von verschiedenen Umfragen und Beliebtheits-Indizes 
analysiert werden. Dabei darf allerdings nicht außer
Acht gelassen werden, dass es sich bei VS Code und
IntelliJ nicht zwingend um Konkurrenzprodukte handelt,
sondern die beiden IDEs unterschiedliche Zielgruppen ansprechen.
Natürlich ist auch zu beachten wie viele Plugins 
für die Entwicklungsumgebung existieren und somit, wie 
populär die Entwicklungsumgebung bei Plugin EntwicklerInnen ist. 

\section{Performance} 
\label{sec:Kriterien_Performance}    
Da es sich bei den IDEs um komplexe Systeme handelt und die 
entwickelten Plugins sehr eng mit diesen Systemen verwoben sind,
ist es nur schwer möglich, die Performance der entwickelten
Plugins zu vergleichen. Allerdings kann die Performance 
der IDEs, inklusive der verwendeten Plugins, verglichen werden.

\section{Feature-Umfang} 
\label{sec:Kriterien_FeatureUmfang}    
Hier soll verglichen werden, wie viele Funktionen
den EntwicklerInnen von den Plugin APIs 
bereits zur Verfügung gestellt werden.

\section{Einfachheit der Verwendung der API} 
\label{sec:Kriterien_Intuitivität}    
Die Einfachheit und Intuitivität der Plugin APIs kann bewertet werden,
indem verglichen wird, wie leicht es ist, die
angebotenen Features zu verwenden.
Dabei muss unter anderem beachtet werden wie viel Vorwissen über die
API man mitbringen muss, wie viele Zusammenhänge es
gibt, die nicht von Anfang an klar sind und ob es 
häufige Fallstricke gibt, die EntwicklerInnen vermeiden müssen.

\section{Dokumentation der API} 
\label{sec:Kriterien_Dokumentation}    
Um eine Plugin API zu erlernen, dient als erste Grundlage
immer die Dokumentation. Hier muss vor allem
auf Übersichtlichkeit und Vollständingkeit großer Wert
gelegt werden.

\section{Testbarkeit des Plugins} 
\label{sec:Kriterien_Testbarkeit}
Um sicherzustellen, dass bei NutzerInnen keine Fehler auftreten,
müssen entwickelte Plugins getestet werden.
Dafür sollten die Plugin APIs das automatisierte
Testen der Plugins nicht nur ermöglichen, sondern möglichst
gut unterstützen.

\section{Möglichkeiten des Publishings} 
\label{sec:Kriterien_Publishing}    
Hier soll vor allem betrachtet werden, wie
einfach es für EntwicklerInnen ist, ein Plugin zu 
veröffentlichen, aber auch welche Möglichkeiten es gibt, 
das Plugin zu vermarkten. Dabei ist nicht nur die Möglichkeit
auf Monetarisierung relevant, sondern auch die
Optionen für die Darstellung des Plugins im Marketplace 
(z.B. Einbindung von Logos oder Beschreibungen).

\section{Installationsprozess des Plugins} 
\label{sec:Kriterien_Installationsprozess}    
Zuletzt ist auch wichtig, wie leicht es für potentielle 
NutzerInnen ist, das entwickelte Plugin zu finden und
in ihrer Programmierumgebung zu installieren.


%\begin{description}
%     \item[Popularität der Entwicklungsumgebung] 
%     \label{sec:Kriterien_Popularität}    
%     Die Popularität von VS Code und IntelliJ kann aufgrund
%     von verschiedenen Umfragen und Beliebtheits-Indizes 
%     analysiert werden. Dabei darf allerdings nicht außer
%     Acht gelassen werden, dass es sich bei VS Code und
%     IntelliJ nicht zwingend um Konkurrenzprodukte handelt,
%     sondern die beiden IDEs unterschiedliche Zielgruppen ansprechen.
%     Natürlich ist auch zu beachten wie viele Plugins 
%     für die Entwicklungsumgebung existieren und somit, wie 
%     populär die Entwicklungsumgebung bei Plugin EntwicklerInnen ist. 

%     \item[Performance] 
%     \label{sec:Kriterien_Performance}    
%     Da es sich bei den IDEs um komplexe Systeme handelt und die 
%     entwickelten Plugins sehr eng mit diesen Systemen verwoben sind,
%     ist es nur schwer möglich, die Performance der entwickelten
%     Plugins zu vergleichen. Allerdings kann die Performance 
%     der IDEs, inklusive der verwendeten Plugins, verglichen werden.

%     \item[Feature-Umfang] 
%     \label{sec:Kriterien_FeatureUmfang}    
%     Hier soll verglichen werden, wie viele Funktionen
%     den EntwicklerInnen von den Plugin APIs 
%     bereits zur Verfügung gestellt werden.

%     \item[Einfachheit der Verwendung der API] 
%     \label{sec:Kriterien_Intuitivität}    
%     Die Intuitivität der Plugin APIs kann bewertet werden,
%     indem verglichen wird, wie einfach es ist, die
%     angebotenen Features zu verwenden.
%     Dabei muss unter anderem beachtet werden wie viel Vorwissen über die
%     API man mitbringen muss, wie viele Zusammenhänge es
%     gibt, die nicht von Anfang an klar sind und ob es 
%     häufige Fallstricke gibt, die EntwicklerInnen vermeiden müssen.

%     \item[Dokumentation der API] 
%     \label{sec:Kriterien_Dokumentation}    
%     Um eine Plugin API zu erlernen, dient als erste Grundlage
%     immer die Dokumentation. Hier muss vor allem
%     auf Übersichtlichkeit und Vollständingkeit großer Wert
%     gelegt werden.

%     \item[Testbarkeit des Plugins] 
%     \label{sec:Kriterien_Testbarkeit}
%     Um sicherzustellen, dass bei NutzerInnen keine Fehler auftreten,
%     müssen entwickelte Plugins getestet werden.
%     Dafür sollten die Plugin APIs das automatisierte
%     Testen der Plugins nicht nur ermöglichen, sondern möglichst
%     gut unterstützen.

%     \item[Möglichkeiten des Publishings] 
%     \label{sec:Kriterien_Publishing}    
%     Hier soll vor allem betrachtet werden, wie
%     einfach es für EntwicklerInnen ist, ein Plugin zu 
%     veröffentlichen, aber auch welche Möglichkeiten es gibt, 
%     das Plugin zu vermarkten. Dabei ist nicht nur die Möglichkeit
%     auf Monetarisierung relevant, sondern auch die
%     Optionen für die Darstellung des Plugins im Marketplace 
%     (z.B. Einbindung von Logos oder Beschreibungen).

%     \item[Installationsprozess des Plugins] 
%     \label{sec:Kriterien_Installationsprozess}    
%     Zuletzt ist auch wichtig, wie leicht es für potentielle 
%     NutzerInnen ist, das entwickelte Plugin zu finden und
%     in ihrer Programmierumgebung zu installieren.

% \end{description}