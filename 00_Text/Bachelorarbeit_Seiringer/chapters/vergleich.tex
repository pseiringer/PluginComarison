\chapter{Vergleich der Kriterien}
\label{cha:Vergleich}

\section{Popularität der Entwicklungsumgebung}
\label{sec:Vergleich_Popularität}

\subsection{Visual Studio Code}

Laut der Stack Overflow Developer Survey von 2023 ist
VS Code der große Spitzenreiter der IDEs. Es wurde von
73,71 Prozent der EntwicklerInnen angegeben,
dass sie VS Code verwenden \cite{StackOverflowSurvey2023}.
Auch der PYPL Index bestätigt die Beliebtheit von VS Code \cite{PYPL}.
Dieser Index misst die Anzahl von Suchanfragen in der Google Suchmaschine
und reiht VS Code auf dem zweiten Platz, gleich hinter Visual Studio.

Die Anzahl von Extensions die im Visual Studio Marketplace angeboten werden
beträgt aktuell über 54.000 \cite{VSCodeMarketplace}. Allerdings sind 
darunter auch viele Extensions die nicht viel verwendet werden und die nur
wenige downloads vorweisen können. Zählt man nur die größten 
Extensions, die über eine Million Downloads erreicht haben,
kommt man auf etwas über 350 Extensions.

\subsection{IntelliJ IDEA}

In der Stack Overflow Developer Survey gaben 2023
26,42 Prozent der EntwicklerInnen an, IntelliJ IDEA zu verwenden,
womit es in der Umfrage den dritten Platz erreichte \cite{StackOverflowSurvey2023}.
PYPL stuft IntelliJ auf Platz sieben ihres Index ein \cite{PYPL}.
Allerdings darf hierbei nicht vergessen werden, dass Plugins
für die IntelliJ Platform auch in anderen JetBrains IDEs installiert
werden können. Berechnet man aus den Daten der Stack Overflow Umfrage
eine gemeinsame Beliebtheit der IntelliJ Platform IDEs, so findet man,
dass 52,87 Prozent der Entwicklerinnen eine solche Programmierumgebung nutzen \cite{StackOverflowSurvey}.
Weiters muss beachtet werden, dass die Zielgruppe von IntelliJ IDEA
vor allem Java EntwicklerInnen sind. Bei Umfragen mit rein Java
EntwicklerInnen schneidet IntelliJ IDEA meist mit dem ersten Platz ab \cite{JRebelIDEs,JRebelDeveloperProductivityReport,BetterprojectsfasterPouplarityIndex}.

Im JetBrains Marketplace werden für die gesamte IntelliJ Platform aktuell
etwas über 7.800 Plugins zum Download angeboten. Zählt man nur die 
großen Plugins mit mehr als einer Million Downloads, kommt man auf
knapp über 100 Plugins.

\subsection{Vergleich}

Während es sich bei VS Code um einen allgemeinen Code-Editor handelt,
der sehr vielseitig eingesetzt werden kann, bietet die IntelliJ Platform
eher spezifische Werkzeuge in erster Linie auf die Entwicklung in
einer einzelnen Programmiersprache ausgerichtet sind.
Obwohl VS Code zwar insgesamt der häufiger genutzte Editor ist,
sind die JetBrains IDEs für das Programmieren in einer 
speziellen Sprache (zum Beispiel IntelliJ IDEA für Java) oft die
beliebtere Wahl.

% //TODO maybe add a table
% \begin{tabular}{@{}llll@{}}
%     \toprule
%     VS Code & IntelliJ Platform \\
%     \midrule
%     C++ & Kompiliert & Applikationen & ISO/IEC 14882:2020 \\
%     COBOL & Kompiliert & Business & ISO/IEC 1989:2014 \\
%     JavaScript & Interpretiert & Web & ECMA-262 \\
%     Python & Interpretiert & Machine Learning & PEPs \\
%     \bottomrule
% \end{tabular}


    

\section{Performance}
\label{sec:Vergleich_Performance}

\subsection{Visual Studio Code}

Mithilfe der Anwendung \emph{AppTimer} von PassMark konnte
die durchschnittliche Startzeit von VS Code ermittelt werden \cite{PassMarkAppTimer}.
AppTimer wurde so konfiguriert, dass VS Code automatisch gestartet
und wieder geschlossen wurden. Beim Start wurde von VS Code ein
einfacher Projektordner geladen. Der AppTimer maß bei jedem Start
die Zeit die VS Code brauchte, um in einen Zustand zu gelangen,
in dem Nutzereingaben angenommen werden konnten.
Diese Messung wurde in 100 Iterationen wiederholt.
Daraus ergab sich eine durchschnittliche Startzeit von 0.294 Sekunden.

Die Hardwareanforderungen sind sehr gering. VS Code benötigt weniger
als 500 MB Festplattenspeicher, 1 GB RAM und eine Prozessorgeschwindigkeit
von 1.6 GHz um lauffähig zu sein.

\subsection{IntelliJ IDEA}

Die Messungen mithilfe der AppTimer-Anwendung wurden auch
für IntelliJ IDEA Ultimate in 100 Iterationen wiederholt.
Hier war die durchschnittliche Startzeit bei 8.123 Sekunden.

Als Hardwareanforderungen benötigt IntelliJ IDEA
mindestens 3.5 GB Festplattenspeicher, 2 GB RAM und eine
moderne CPU.

\subsection{Vergleich}

Da es sich bei VS Code nur um einen leichtgewichtigen Editor handelt,
hat dieser auch einen klaren Vorteil bei der Performance 
und den Hardwareanforderungen.
IntelliJ IDEA ist eine vollständige IDE und hat daher auch längere
Startzeiten und höhere Hardwareanforderungen.
Auf einigermaßen modernen Geräten sollten beide Entwicklungsumgebungen
problemlos ausführbar und verwendbar sein. Nur bei starken
Hardwarelimitierungen kann VS Code hier einen Vorteil bieten.


\section{Feature Umfang}
\label{sec:Vergleich_FeatureUmfang}

\subsection{Visual Studio Code}

\subsection{IntelliJ IDEA}

\subsection{Vergleich}


\section{Intuitivität der API}
\label{sec:Vergleich_Intuitivität}

\subsection{Visual Studio Code}

\subsection{IntelliJ IDEA}

\subsection{Vergleich}


\section{Dokumentation der API}
\label{sec:Vergleich_Dokumentation}

\subsection{Visual Studio Code}

\subsection{IntelliJ IDEA}

\subsection{Vergleich}


\section{Testbarkeit des Plugins}
\label{sec:Vergleich_Testbarkeit}

\subsection{Visual Studio Code}

\subsection{IntelliJ IDEA}

\subsection{Vergleich}


\section{Möglichkeiten des Publishings}
\label{sec:Vergleich_Publishing}

\subsection{Visual Studio Code}

\subsection{IntelliJ IDEA}

\subsection{Vergleich}


\section{Installationsprozess des Plugins}
\label{sec:Vergleich_Installationsprozess}

\subsection{Visual Studio Code}

\subsection{IntelliJ IDEA}

\subsection{Vergleich}
