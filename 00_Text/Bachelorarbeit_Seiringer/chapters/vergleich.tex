\chapter{Vergleich anhand der Bewertungskriterien}
\label{cha:Vergleich}

\section{Popularität der Entwicklungsumgebung}
\label{sec:Vergleich_Popularität}

\subsection{Visual Studio Code}

Laut der Stack Overflow Developer Survey von 2023 ist
VS Code der große Spitzenreiter der IDEs. Es wurde von
73,71\% der EntwicklerInnen angegeben,
dass sie VS Code verwenden \cite{StackOverflowSurvey2023}.
Auch der PYPL Index bestätigt die Beliebtheit von VS Code~\cite{PYPL}.
Dieser Index misst die Anzahl von Suchanfragen in der Google Suchmaschine
und reiht VS Code auf den zweiten Platz hinter Visual Studio.

Die Anzahl von Extensions, die im Visual Studio Marketplace angeboten werden,
beträgt aktuell über 54.000 \cite{VSCodeMarketplace}. Allerdings sind 
darunter auch viele Extensions, die nicht viel verwendet werden und die nur
wenige Downloads vorweisen können. Zählt man nur die größten 
Extensions, die über eine Million Downloads erreicht haben,
kommt man auf etwas über 350 Extensions.

\subsection{IntelliJ IDEA}

In der Stack Overflow Developer Survey gaben 2023
26,42\% der EntwicklerInnen an, IntelliJ IDEA zu verwenden,
womit es in der Umfrage den dritten Platz erreichte~\cite{StackOverflowSurvey2023}.
PYPL stuft IntelliJ auf Platz sieben ihres Index ein \cite{PYPL}.
Allerdings darf hierbei nicht vergessen werden, dass Plugins
für die IntelliJ Platform auch in anderen JetBrains IDEs installiert
werden können. Berechnet man aus den Daten der Stack Overflow Umfrage
eine gemeinsame Beliebtheit der IntelliJ Platform IDEs, so findet man,
dass 52,87\% der Entwicklerinnen eine solche Programmierumgebung nutzen \cite{StackOverflowSurvey}.
Weiters muss beachtet werden, dass die Zielgruppe von IntelliJ IDEA
vor allem Java EntwicklerInnen sind. Bei Umfragen mit Java-EntwicklerInnen 
schneidet IntelliJ IDEA meist mit dem ersten Platz ab \cite{JRebelIDEs,JRebelDeveloperProductivityReport,BetterprojectsfasterPouplarityIndex}.

Im JetBrains Marketplace werden für die gesamte IntelliJ Platform aktuell
etwas über 7.800 Plugins zum Download angeboten. Zählt man nur die 
großen Plugins mit mehr als einer Million Downloads, kommt man auf
knapp über 100 Plugins.

\subsection{Vergleich}

Während es sich bei VS Code um einen allgemeinen Code-Editor handelt,
der sehr vielseitig eingesetzt werden kann, bietet die IntelliJ-Plattform
eher spezifische Werkzeuge, die in erster Linie auf die Entwicklung in
einer einzelnen Programmiersprache ausgerichtet sind.
Obwohl VS Code zwar insgesamt der häufiger genutzte Editor ist,
sind die JetBrains IDEs für das Programmieren in einer 
speziellen Sprache (zum Beispiel IntelliJ IDEA für Java) oft die
beliebtere Wahl.
    

\section{Performance}
\label{sec:Vergleich_Performance}

\subsection{Visual Studio Code}

Mithilfe der Anwendung \emph{AppTimer} von PassMark konnte
die durchschnittliche Startzeit von VS Code ermittelt werden \cite{PassMarkAppTimer}.
AppTimer wurde so konfiguriert, dass VS Code automatisch gestartet
und wieder geschlossen wird. Beim Start wurde von VS Code ein
einfacher Projektordner geladen. Der AppTimer maß bei jedem Start
die Zeit die VS Code brauchte, um in einen Zustand zu gelangen,
in dem Nutzereingaben angenommen werden konnten.
Diese Messung wurde in 100 Iterationen wiederholt.
Daraus ergab sich eine durchschnittliche Startzeit von 0.294 Sekunden.

Die Hardwareanforderungen sind gering. VS Code benötigt weniger
als 500 MB Festplattenspeicher, 1 GB RAM und eine Prozessorgeschwindigkeit
von 1.6 GHz um lauffähig zu sein.

\subsection{IntelliJ IDEA}

Die Messungen mithilfe der AppTimer-Anwendung wurden auch
für IntelliJ IDEA Ultimate in 100 Iterationen wiederholt.
Hier war die durchschnittliche Startzeit bei 8.123 Sekunden.

Als Hardwareanforderungen benötigt IntelliJ IDEA
mindestens 3.5 GB Festplattenspeicher, 2 GB RAM und eine
moderne CPU.

\subsection{Vergleich}

Da es sich bei VS Code nur um einen leichtgewichtigen Editor handelt,
hat dieser auch einen klaren Vorteil bei der Performance 
und den Hardwareanforderungen.
IntelliJ IDEA ist eine vollständige IDE und hat daher auch längere
Startzeiten und höhere Hardwareanforderungen.
Auf einigermaßen modernen Geräten sollten beide Entwicklungsumgebungen
problemlos ausführbar und verwendbar sein. Allerdings kann VS Code hier
einen Vorteil bieten, wenn gleichzeitig andere, ressourcenintensive Programme
auf dem Rechner laufen, oder Hardwarelimitierungen vorliegen.


\section{Feature Umfang}
\label{sec:Vergleich_FeatureUmfang}

\subsection{Visual Studio Code}

VS Code unterstützt viele grundlegende Features (wie zum Beispiel 
Kommandos, Interaktionen mit dem Editor, Einstellungen oder persistente
Datenspeicherung) sehr gut. Auch werden komplexe Spracherweiterungen
durch den Einsatz des \emph{Language Server Protocol} möglich.
Weiters bietet VS Code auch dezidierte Schnittstellen an, wenn Extensions
den Debugger oder das Testsystem von VS Code erweitern möchten.
Zur Anzeige bietet VS Code ein \emph{Tree View} und ein 
\emph{Webview}, in welchem beliebige Inhalte dargestellt werden können.

\subsection{IntelliJ IDEA}

Auch die IntelliJ Platform bietet viele grundlegende Funktionalitäten
für Plugins an (zum Beispiel Aktionen, Interaktion mit den geöffneten Dokumenten
und Projekten oder Einstellungen). Die Besonderheit, die IntelliJ
in Bezug auf Spracherweiterungen
mitbringt, ist das \emph{Program Structure Interface (PSI)}. 
Auf diese Weise können Sprachen gut in die IDEs integriert werden
und es kann extrem effizient mit dem Code interagiert werden.
Die Einbindung eines Debuggers oder Compilers sollte zwar grundsätzlich
möglich sein, allerdings gibt es hierzu (aktuell noch) keine Dokumentation.
Zur Anzeige wird auf Java Swing \emph{Tool Windows} gesetzt.
Weiters ist es möglich, in IntelliJ Plugins eigene 
\emph{Extension Points} zu deklarieren, die von weiteren Plugins
wiederum erweitert werden können.

\subsection{Vergleich}

Beide APIs sind sich in den angebotenen Grundlagen sehr ähnlich.
Der größte Unterschied liegt bei den Spracherweiterungen, da VS Code 
vollständig auf das Language Server Protocol setzt, während IntelliJ
dieses nicht verwendet. Natürlich kann es für Plugin EntwicklerInnen
auch einen Unterschied machen, ob sie Benutzerschnittstellen
lieber mit HTML + CSS + JavaScript (Webviews) oder
mit Java Swing (Tool Windows) implementieren. 
Ein großer Vorteil den IntelliJ IDEA in Bezug auf Feature Umfang
bietet, ist das Erweiterungssystem mittels Extension Points.


\section{Einfachheit der Verwendung der API}
\label{sec:Vergleich_Intuitivität}

\subsection{Visual Studio Code}

Mit der VS Code Extension API wird ausschließlich
über das \emph{Extension Manifest} und das Modul \emph{vscode} interagiert.
Dieses Modul fasst alle Code-Schnittstellen zusammen und ist
in übersichtliche Abschnitte (z.B. commands, window, workspace) unterteilt.

\subsection{IntelliJ IDEA}

In IntelliJ werden Deklarationen im \emph{Plugin Configuration File} gemacht.
Die Schnittstellen, die vom Plugin implementiert werden müssen, und
die Klassen, mit denen interagiert wird, hängen von den deklarierten
\emph{Extension Points} ab. Es handelt sich dabei um unterschiedlichste
Klassen aus dem Quellcode der IntelliJ Platform.

\subsection{Vergleich}

Die VS Code API ist sehr übersichtlich und intuitiv und kann auch
schon nach kurzem Einlesen in die Dokumentation verwendet werden.
IntelliJ Plugins müssen mit vielen verschiedenen Klassen interagieren
bei, denen auf verschiedene Implementierungsdetails geachtet werden muss.
Dadurch kann auch die Dokumentation auf den ersten Blick sehr komplex
wirken. Allerdings sind IntelliJ Plugins in ihren Möglichkeiten der
Interaktion mit der API weniger stark eingeschränkt.


\section{Dokumentation der API}
\label{sec:Vergleich_Dokumentation}

\subsection{Visual Studio Code}

Die Dokumentation für VS Code Extensions ist sehr übersichtlich.
In den ersten Abschnitten werden Grundlagen und häufig 
verwendete Features beschrieben.
In weiteren Abschnitten gibt es Anleitungen für spezielle Features,
Dokumentation für Spracherweiterungen und Beschreibungen
für das Testen und Publishen von Extension.
Weiters behandelt die Dokumentation auch eine vollständige Beschreibung
aller möglichen Interaktionen mit dem Extension Manifest
und eine vollständige Beschreibung der VS Code API Schnittstelle.
Auch ein Repository mit einfachen Code-Beispielen ist in der Dokumentation verlinkt.

\subsection{IntelliJ IDEA}

Die IntelliJ-Dokumentation beschreibt in den ersten Abschnitten kurz
die Grundlagen. Dabei werden vor allem im
Abschnitt \emph{Base Platform} viele häufig genutzte Funktionalitäten
für Plugins beschrieben. Die nächsten Abschnitte beschreiben
meist ein System der IntelliJ-Plattform und wie ein Plugin mit diesem
System interagieren kann. Zum Beispiel gibt es Abschnitte für \emph{Project Model}
oder \emph{PSI}. Leider findet man immer wieder Abschnitte, die mit der 
Bemerkung \enquote{Will be available soon} ausgegraut sind.
Es gibt zwar eine Liste von möglichen Extension Points, allerdings
finden sich in dieser keine Beschreibungen. Anstatt einer vollständigen
Dokumentation wird auf den IntelliJ-Plattform-Quellcode verwiesen.
Allerdings lassen auch die Kommentare im Code sehr zu wünschen übrig.
Auch für IntelliJ gibt es ein Repository mit Code-Beispielen, die sehr
hilfreich sein können.

\subsection{Vergleich}

VS Code bietet eine vollständige, übersichtliche und gut geschriebene 
Dokumentation für Plugin-EntwicklerInnen. Die IntelliJ-Dokumentation
bietet einen guten Überblick über die Systeme der IntelliJ-Plattform,
allerdings ist sie eher unübersichtlich und teilweise 
(an kritischen Stellen) unvollständig.
Beide Dokumentationen bieten Code-Beispiele an.

\section{Testbarkeit des Plugins}
\label{sec:Vergleich_Testbarkeit}

\subsection{Visual Studio Code}

VS Code erlaubt das Testen der Plugins und ermöglicht
dabei Zugriff auf die Schnittstellen der Plugin API.
Integrationstests werden mithilfe einer VS-Code-Instanz ausgeführt.

\subsection{IntelliJ IDEA}

IntelliJ ermöglicht das Testen des Plugins, in dem
von bestimmten Basis-Testklassen abgeleitet wird. Durch diese
Basisklassen werden zusätzliche Hilfsmethoden für die Tests
angeboten. Die Integrationstests werden in einer
headless-Instanz von IntelliJ IDEA ausgeführt, die großteils
einer echten Instanz von IntelliJ IDEA entspricht.

\subsection{Vergleich}

In IntelliJ müssen zwar einige Implementierungsdetails in den Testklassen
beachtet werden, dafür bietet IntelliJ eine bessere Unterstützung
durch Hilfsmethoden für Tests.


\section{Möglichkeiten des Publishings}
\label{sec:Vergleich_Publishing}

\subsection{Visual Studio Code}

Das Hochladen und Verwalten von Plugins im Marketplace ist
einfach und ausführlich dokumentiert. Das Werkzeug \emph{vsce}
bietet Möglichkeiten zum Verwalten von Extensions und 
erlaubt das automatisierte Publishing mittels CI/CD Pipelines.
Auf der Plugin-Seite des Marketplace kann eine \emph{Markdown}-Datei
angegeben werden, in der unter anderem auch Bilder oder Aufzählungen 
eingebunden und dargestellt werden können. 
Weiters bietet der Marketplace eine Seite für Reviews
und eine Q\&A-Seite für jedes Plugin.

\subsection{IntelliJ IDEA}

Da bei IntelliJ zu Beginn eine Signatur erstellt werden muss, ist das
erste (manuelle) Hochladen etwas aufwändiger. Allerdings
gibt es auch hier die Möglichkeit zum automatischen Publishing.
Die Plugin-Beschreibungen können in HTML formatiert werden. Es gibt
zusätzliche Felder um Bildschirmaufnahmen, ein Vorschauvideo oder
weitere Seiten in Markdown-Formattierung zur Marketplace-Seite
hinzuzufügen. Weiters gibt es die Möglichkeit kurze Beschreibungskarten
des Plugins erstellen zu lassen, die dann zum Beispiel auf 
einer eigenen Webseite eingebunden werden können.

\subsection{Vergleich}

Beide IDEs bieten einen ausgezeichneten Marketplace an
und ermöglichen das einfache Veröffentlichen eines Plugins.
Zur Darstellung des Plugins im Marketplace gibt es bei JetBrains
jedoch mehr Optionen.


\section{Installationsprozess des Plugins}
\label{sec:Vergleich_Installationsprozess}

\subsection{Visual Studio Code}

Die Installation des Plugins bei NutzerInnen funktioniert
über den Marketplace. Dieser ist in VS Code stark in den 
Editor eingebunden und es kann einfach nach 
Extensions gesucht werden. Das Installieren einer gewünschten
Extension funktioniert also über wenige Klicks.
Alternativ kann das Plugin auch auf der Marketplace Webseite
gesucht und von dort aus installiert werden.
Möchte man den VS Code Marketplace nicht nutzen, gibt es auch
die Möglichkeit, die Extension als \emph{.vsix}-Datei
zu verpacken und diese manuell von der Festplatte aus zu installieren.

\subsection{IntelliJ IDEA}

In IntelliJ IDEA kann ein Plugin über den JetBrains Marketplace
installiert werden. Der Marketplace ist dabei in den Einstellung
der IDE eingebunden. Auch hier kann das gewünschte Plugin einfach
gesucht und installiert werden.
Besucht man die Marketplace Webseite, so gibt es bei dieser nur einen
Download-Knopf, mit welchem das Plugin als \emph{.zip}-Datei heruntergeladen
werden kann. Diese gezippten Plugins können auch
unabhängig vom Marketplace erstellt und manuell
von der Festplatte aus installiert werden.

\subsection{Vergleich}

Beide Plattformen bieten zur Installation die selben Möglichkeiten.
Allerdings braucht man in VS Code einen Klick weniger, um eine
Extension zu finden, da der Marketplace nicht in den Einstellungen
versteckt ist, wie in IntelliJ.


\section{Übersicht}

Die Tabelle \ref{tab:Ueberblick_Kriterien} zeigt noch
einmal eine Übersicht über die Bewertungskriterien,
deren Vergleich in den vorherigen Abschnitten genauer
ausgeführt wurde. Ein Plus steht hierbei für die Entwicklungsumgebung,
die bei dem Kriterium mehr Vorteile bietet.

\begin{table}[hb]
    \caption{Überblick über die Bewertungen.}
    \label{tab:Ueberblick_Kriterien}
    \centering
    \small % Reduce font size
    \setlength{\tabcolsep}{10pt} % separator between columns (standard = 6pt)
    \renewcommand{\arraystretch}{1.25} % vertical stretch factor (standard = 1.0)

    \begin{tabular}{@{}llll@{}}
        \toprule
         & VS Code & IntelliJ Platform \\
        \midrule
        Popularität der Entwicklungsumgebung & + &  \\
        Performance & + &  \\
        Feature Umfang &  & + \\
        Intuitivität der API & + &  \\
        Dokumentation der API & + &  \\
        Testbarkeit des Plugins &  & + \\
        Möglichkeiten des Publishings &  & + \\
        Installationsprozess des Plugins & + &  \\
        \bottomrule
    \end{tabular}
    
\end{table}