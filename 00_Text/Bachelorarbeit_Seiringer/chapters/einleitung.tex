\chapter{Einleitung}
\label{cha:Einleitung}


\section{Motivation}
\label{sec:Motivation}

SoftwareentwicklerInnen arbeiten täglich mit verschiedensten 
Werkzeugen und Entwicklungsumgebungen, sogenannten IDEs 
(=Integrated Development Environment). Diese Plattformen 
bieten teils sehr unterschiedliche Funktionalitäten, die 
die Softwareentwicklung erleichtern sollen. Dabei bieten 
sie Unterstützung für verschiedenste Programmiersprachen 
und Technologien und binden zahlreiche Werkzeuge für 
spezifische Anwendungsfälle ein. Aufgrund des immer rascher
werdenden Entstehens von neuen Technologien bieten mehr 
und mehr IDEs Möglichkeiten zur Entwicklung von eigenen 
Plugins, welche dann auch an andere EntwicklerInnen 
bereitgestellt werden können. So können in kürzester 
Zeit neue Technologien unterstützt werden und 
EntwicklerInnen haben selbst die Macht darüber zu 
entscheiden welche Plugins sie nutzen möchten und 
welche nicht.

Vor der Entwicklung solcher Plugins ist es wichtig 
zu entscheiden für welche IDE das Plugin erstellt 
werden soll. Dabei spielen Aspekte wie zum Beispiel 
die Einfachheit und Flexibilität in der Entwicklung, 
der Umfang an angebotener Funktionalität, die Möglichkeit 
die Nutzerinteraktion und somit die User Experience 
zu steuern und viele weitere eine Rolle. 
Diese Bachelorarbeit versucht in diesen Bereichen 
einen Überblick zu schaffen und vergleicht hierfür 
die Plugin Entwicklung in zwei der momentan beliebtesten 
IDEs, Visual Studio Code und IntelliJ IDEA. Durch den 
Vergleich der beiden Produkte und dem Herausarbeiten 
und Aufbereiten der Unterschiede wird es anderen 
EntwicklerInnen erleichtert diese Entscheidung zu treffen.%
\cite{ArnoldKen1996TJpl,HagosTed2022BII:,KurbatovaZarina2021TIPa,StackOverflowSurvey2023}


\section{Ziel}
\label{sec:Ziel}

\section{Aufbau}
\label{sec:Aufbau}