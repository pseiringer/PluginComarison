\chapter{Einleitung}
\label{cha:Einleitung}


\section{Motivation}
\label{sec:Motivation}

SoftwareentwicklerInnen arbeiten täglich mit verschiedensten 
Werkzeugen und integrierten Entwicklungsumgebungen 
(engl. Integrated Development Environment, IDE). Entwicklungsumgebungen 
bieten teils sehr unterschiedliche Funktionalitäten, die 
die Softwareentwicklung erleichtern sollen. Dabei bieten 
sie Unterstützung für verschiedenste Programmiersprachen 
und Technologien und binden zahlreiche Werkzeuge für 
spezifische Anwendungsfälle ein. Aufgrund des immer rascher
werdenden Entstehens von neuen Technologien bieten
IDEs Möglichkeiten zur Entwicklung von eigenen 
Plugins, welche dann auch an andere EntwicklerInnen 
bereitgestellt werden können. So können in kürzester 
Zeit neue Technologien unterstützt werden und 
EntwicklerInnen haben selbst die Freiheit darüber zu 
entscheiden, welche Plugins sie nutzen möchten.

Vor der Entwicklung solcher Plugins ist es wichtig 
zu entscheiden, für welche IDE das Plugin erstellt 
werden soll. Dabei spielen unter anderem Aspekte wie 
die Einfachheit und Flexibilität in der Entwicklung, 
der Umfang an angebotener Funktionalität und die Möglichkeit
die Nutzerinteraktion und somit die User Experience 
zu steuern eine Rolle. Aufgrund der hohen 
Komplexität der IDEs und der damit einhergehenden 
Komplexität der Plugin-Schnittstellen, kann es für EntwicklerInnen
ohne Erfahrung in der Plugin-Entwicklung allerdings schwierig sein,
diese Entscheidung zu treffen.

\section{Ziel}
\label{sec:Ziel}

Diese Bachelorarbeit vergleicht die Plugin-Entwicklung 
für Visual Studio Code und IntelliJ IDEA. 
Durch den Vergleich der beiden Produkte 
und das Herausarbeiten und Aufbereiten der Unterschiede 
sowie der Gemeinsamkeiten, wird es anderen 
EntwicklerInnen leichter möglich, eine Entscheidung
bei der Wahl der Entwicklungsumgebung zu treffen.
Durch das gewonnene Wissen soll auch der Einstieg
in die Entwicklung für die jeweilige IDE erleichtert werden.

\section{Aufbau}
\label{sec:Aufbau}

In Kapitel \ref{cha:Grundlagen} wird das grundlegende 
Wissen erarbeitet, welches für die Entwicklung von Plugins 
beziehungsweise Erweiterungen in VS Code und IntelliJ IDEA nötig ist.
Um das erlangte Wissen praktisch anzuwenden, wird in
den Kapiteln \ref{cha:Prototyp}, \ref{cha:EntwicklungVsCode} und \ref{cha:EntwicklungIntelliJ} 
ein Prototyp auf beiden Plattformen erstellt.
In den Kapiteln \ref{cha:Kriterien} und \ref{cha:Vergleich} wird ein Katalog 
an Bewertungskriterien aufgebaut, auf deren Basis 
das Arbeiten mit den Systemen dann verglichen wird.