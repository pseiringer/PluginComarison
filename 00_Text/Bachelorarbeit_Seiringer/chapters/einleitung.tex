\chapter{Einleitung}
\label{cha:Einleitung}


\section{Motivation}
\label{sec:Motivation}

SoftwareentwicklerInnen arbeiten täglich mit verschiedensten 
Werkzeugen und Entwicklungsumgebungen, sogenannten IDEs 
(=Integrated Development Environment). Diese Plattformen 
bieten teils sehr unterschiedliche Funktionalitäten, die 
die Softwareentwicklung erleichtern sollen. Dabei bieten 
sie Unterstützung für verschiedenste Programmiersprachen 
und Technologien und binden zahlreiche Werkzeuge für 
spezifische Anwendungsfälle ein. Aufgrund des immer rascher
werdenden Entstehens von neuen Technologien bieten mehr 
und mehr IDEs Möglichkeiten zur Entwicklung von eigenen 
Plugins, welche dann auch an andere EntwicklerInnen 
bereitgestellt werden können. So können in kürzester 
Zeit neue Technologien unterstützt werden und 
EntwicklerInnen haben selbst die Macht darüber zu 
entscheiden, welche Plugins sie nutzen möchten und 
welche nicht.

Vor der Entwicklung solcher Plugins ist es wichtig 
zu entscheiden, für welche IDE das Plugin erstellt 
werden soll. Dabei spielen unter anderem Aspekte wie 
die Einfachheit und Flexibilität in der Entwicklung, 
der Umfang an angebotener Funktionalität und die Möglichkeit
die Nutzerinteraktion und somit die User Experience 
zu steuern eine Rolle. Aufgrund der hohen 
Komplexität der IDEs und der damit einhergehenden 
Komplexität der Plugin-Schnittstellen, kann es für EntwicklerInnen
ohne Erfahrung in der Plugin Entwicklung allerdings schwer sein
diese Entscheidung zu treffen.

\section{Ziel}
\label{sec:Ziel}

Diese Bachelorarbeit versucht für neue Plugin EntwicklerInnen
einen Überblick zu schaffen und vergleicht hierfür 
die Plugin Entwicklung in den IDEs Visual Studio Code 
und IntelliJ IDEA. Durch den Vergleich der beiden Produkte 
und dem Herausarbeiten und Aufbereiten der Unterschiede, 
sowie der Gemeinsamkeiten, wird es anderen 
EntwicklerInnen leichter möglich machen eine Entscheidung
bei der Wahl des Entwicklungsumgebung zu treffen.
Weiters wird, durch das gewonnene Basiswissen, auch der Einstieg
in die Entwicklung für die jeweilige Plattform erleichtert.

\section{Aufbau}
\label{sec:Aufbau}

Zu Beginn der Arbeit wird das grundlegende Wissen erarbeitet,
welches für die Entwicklung in VS Code und IntelliJ IDEA nötig ist.
Um das erlangte Wissen praktisch anzuwenden, wird als nächstes ein
demonstrativer Prototyp auf beiden Plattformen erstellt.
Zum Schluss wird ein Katalog an Bewertungskriterien aufgebaut,
auf deren Basis die Arbeit mit den Systemen dann verglichen wird.