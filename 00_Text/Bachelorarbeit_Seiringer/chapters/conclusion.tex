\chapter{Zusammenfassung}
\label{cha:Conclusion}

Der direkte Vergleich der entwickelten Systeme zeigt,
dass beide APIs sehr unterschiedliche Vorzüge haben.

VS Code konnte vor allem durch die intuitive
Schnittstelle überzeugen. Mir fiel die Implementierung
von neuen Features in der VS Code Extension meist sehr
leicht, wohingegen ich mich bei IntelliJ immer wieder
in die Dokumentation einlesen musste, um nicht
Kleinigkeiten in der Implementierung zu übersehen.
Auch die allgemein hohe Beliebtheit von VS Code
ermöglicht es, das Plugin einer sehr breiten Masse 
an potenziellen NutzerInnen zur Verfügung zu stellen.
Insbesondere für EntwicklerInnen ohne Vorerfahrung
in der Plugin-Entwicklung würde ich daher die Arbeit mit
Visual Studio Code empfehlen.

IntelliJ IDEA trumpft vor allem mit speziellen Features
wie der \emph{PSI}-Schnittstelle und den \emph{Extension Points} auf,
die EntwicklerInnen viele Möglichkeiten bieten.
Auch die Popularität der JetBrains IDEs für spezifische 
Programmiersprachen kann ein ausschlaggebender Faktor
sein. Der größte Nachteil, den IntelliJ für mich mitbringt,
war die teils unvollständige und teils unverständliche
Dokumentation. Möchte man ein Plugin entwickeln, das 
für die Arbeit mit einer bestimmten Sprache (wie Java)
ausgerichtet ist, oder braucht das Plugin, um nützlich zu
sein, die Funktionen, die ein vollständiges IDE bietet, so
sollte definitiv IntelliJ bevorzugt werden.

Insgesamt bieten beide IDEs eine ausgezeichnete 
Unterstützung für die Entwicklung von Plugins und
bringen viele grundlegende Features mit, die
ohne großen Aufwand durch Plugins wiederverwendet
werden können.