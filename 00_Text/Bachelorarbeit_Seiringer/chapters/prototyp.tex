\chapter{Anforderungen an den Prototyp}
\label{cha:Prototyp}

Für den direkten Vergleich der beiden Plugin APIs sollte 
in VS Code und in IntelliJ ein möglichst funktionsgleicher 
Prototyp implementiert werden.

Durch diesen Prototyp können die Features der beiden APIs 
demonstriert und bewertet werden. Weiters bietet der Prototyp
eine praxisnahe Methode um sich auch mit dem Arbeitsablauf der
beiden Plattformen, speziell auch dem Veröffentlichen
des Plugins, vertraut zu machen.

\section{Aufbau}
\label{sec:Prototyp_Aufbau}

Um diese Aufgaben möglichst gut zu erfüllen, wurde das
\enquote{RecentChangesPlugin} entworfen. Kurz gesagt handelt
es sich dabei um ein Plugin, welches Text- oder Codeänderungen 
im Editor mitschreibt und die Wiederholung von gleichen oder
ähnlichen Änderungen erleichtert.
Als Änderung wird hierbei das einfache Abändern oder das Ersetzen eines
Wortes durch ein anderes verstanden. 
Das Mitschreiben von komplexeren Veränderungen, zum Beispiel, 
wenn in mehreren Zeilen gleichzeitig Änderungen vorgenommen 
werden, ist nicht Ziel des Prototypen, da dies das Erkennen 
und das Wiederholen der Änderung stark verkomplizieren würde.

\subsection{Bemerken von Textänderungen}

Der Prototyp soll Änderungen in Text- oder Codedateien automatisch
erkennen können um diese mitzuschreiben. Um solche Änderungen
zu erkennen, muss auch erkannt werden, wann eine Änderung 
vollständig abgeschlossen ist. Hierfür
soll ein Debounce-Effekt % //TODO ask if this needs a citation
genutzt werden. Die erkannten Änderungen 
müssen analysiert und zwischengespeichert werden.

\subsection{Anwenden von Änderungen auf Befehl}

Der Prototyp soll zuvor erkannte Änderungen auf Befehl an der
aktuellen Position im Editor anwenden können. Dafür muss zuerst die
aktuelle Textcursor-Position analysiert werden, um das Wort zu finden,
an dem sich der Textcursor befindet. Danach muss in den 
zwischengespeicherten Änderungen ein passender Eintrag gefunden werden,
der auch auf die aktuelle Position angewendet werden kann.
Dabei sollen neuere Änderungen gegenüber älteren bevorzugt werden.

\subsection{Anwenden von Änderungen durch Tastenkombination}

Der Vorgang des Anwenden von Änderungen soll auch durch eine
Tastenkombination auslösbar sein.

\subsection{Vergessen von alten Änderungen}

Um erkannte Änderungen nicht ewig im Zwischenspeicher zu halten,
soll nur eine bestimmte Menge gleichzeitig gespeichert werden 
können. Wird diese Menge überschritten, so soll die älteste Änderung
aus dem Zwischenspeicher entfernt und somit \enquote{vergessen} werden.

\subsection{Automatische Codevervollständigung}

Anhand der kürzlichen Änderungen sollen auch Vorschläge
in der Codevervollständigung angezeigt werden. 
Da eine Änderung ja immer aus einem entfernten Wort 
und einem ersetzenden Wort besteht, liegt es hier nahe
für die Codevervollständigung nur die ersetzenden Worte
vorzuschlagen. Die entfernten Worte werden also ignoriert.

\subsection{Anzeige der Änderungen}

Damit die NutzerInnen einen Überblick über die kürzlichen Änderungen
haben, sollen alle Änderungen, die sich im Zwischenspeicher befinden,
über eine View angezeigt werden können.

\subsection{Einstellungen}

Es soll möglich sein Einstellungen des Plugins festzulegen, welche
auch beim Schließen der Entwicklungsumgebunge persistent bleiben.

Es soll gespeichert werden:
\begin{itemize}
    \item wie viele Änderungen gleichzeitig im 
        Zwischenspeicher gehalten werden.
    \item welche Debounce Zeit für das Erkennen 
        von Änderungen verwendet wird. 
\end{itemize}

\subsection{Tests}

Es soll eine kleine, demonstrative Menge von Unit- und Integrationstests für
den Plugin Code geschrieben werden.