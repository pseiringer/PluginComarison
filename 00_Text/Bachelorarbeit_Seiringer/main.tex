%%% Dateikodierung: UTF-8

%%% Magic Comments zum Setzen der korrekten Parameter in kompatiblen IDEs
% !TeX encoding = utf8
% !TeX program = pdflatex 
% !TeX spellcheck = de_DE
% !BIB program = biber

\documentclass[bachelor,german,smartquotes]{hgbthesis}
% Zulässige Optionen in [..]: 
%    Typ der Arbeit: 'diploma', 'master' (default), 'bachelor', 'internship'
%		 Zusätzlich für ein Thesis-Exposé: 'proposal' (für 'bachelor' und 'master')
%    Hauptsprache: 'german' (default), 'english'
%    Option zur Umwandlung in typografische Anführungszeichen: 'smartquotes'
%    APA Zitierstil: 'apa'
%%%-----------------------------------------------------------------------------

\RequirePackage[utf8]{inputenc} % bei Verw. von lualatex oder xelatex entfernen!

\graphicspath{{images/}}  % Verzeichnis mit Bildern und Grafiken
\logofile{logo}           % Logo-Datei: images/logo.pdf (kein Logo: \logofile{})
\bibliography{references} % Biblatex-Literaturdatei (references.bib)

%%%-----------------------------------------------------------------------------
\begin{document}
%%%-----------------------------------------------------------------------------

%%%-----------------------------------------------------------------------------
% Angaben für die Titelei (Titelseite, Erklärung etc.)
%%%-----------------------------------------------------------------------------

\title{Vergleich der Erweiterungsmöglichkeiten in Visual Studio Code und IntelliJ IDEA}
\author{Philipp Seiringer}
\programname{Software Engineering}

\programtype{Fachhochschul-Bachelorstudiengang}

\placeofstudy{Hagenberg}
\dateofsubmission{2024}{02}{01} % {YYYY}{MM}{DD}

\advisor{Dr. Josef Pichler} % optional

%\strictlicense % restriktive Lizenz anstatt Creative Commons (nicht empfohlen!)

%%%-----------------------------------------------------------------------------
\frontmatter                                       % Titelei (röm. Seitenzahlen)
%%%-----------------------------------------------------------------------------

\maketitle

\chapter{Kurzfassung}

Das Arbeiten mit IDEs kann durch die Entwicklung von 
Plugins oder Extensions verbessert werden. So ermöglichen sie unter
anderem die automatisierung von Abläufen während der Entwicklung oder
die Einbindung von zusätzlichen Werkzeugen in die bereits bestehende
Entwicklungsumgebung. Die Entscheidung,
für welche Programmierumgebung ein Plugin entwickelt werden soll,
ist allerdings nicht trivial. Für diese Entscheidung ist es von Vorteil,
ein Basiswissen über den Aufbau und die Funktionalität der APIs,
sowie über die Entwicklung solcher Plugins und Extensions zu haben.

Es werden daher die Extension-APIs in den IDEs Visual Studio Code
und IntelliJ IDEA analysiert und verglichen. Eingegangen wird
auf den Aufbau und die Funktionalität der APIs, sowie 
auf die Testbarkeit und das Publishing von Plugins.
Für den Vergleich wird ein Katalog an Bewertungskriterien aufgestellt, 
an denen die beiden Systeme gemessen werden können.

Um für den Vergleich eine praktische Grundlage zu schaffen, wird
ein prototypisches Plugin für beide Plattformen entwickelt. Aufgrund
der dabei gemachten Erfahrungen können die Bewertungskritierien
ausgeführt und verglichen werden. Durch das erarbeitete Wissen
kann effektiv eine Entscheidungsbasis für neue Plugin EntwicklerInnen
geschaffen werden und der Einstieg in die Pluginentwicklung 
erleichtert werden.		
\chapter{Abstract}


\begin{english} %switch to English language rules

Working with IDEs can be vastly improved by plugins
or extensions. For example they allow for automation 
of development tasks, or for integration of additional development
tools into the existing development environment.
However, the decision for which IDE
a plugin should be developed, is (especially without any
previous experience) far from trivial. For this decision,
it is advantageous to have a fundamental knowledge of
the functionality of the APIs, as well as of how such
plugins and extensions are developed.

Therefore, the extension APIs in Visual Studio Code and 
IntelliJ IDEA will be analysed and compared. Especially
the structure and functionality of the APIs, but also
their testability and publishing methods will be looked into.
In order to make a comparison, a catalog of criteria will be created,
against which the two systems can be measured.

In order to create a practical basis for the comparison,
a prototypical plugin will be developed for both platforms.
Based on the experiences made while developing this 
prototype, the criteria can be evaluated and compared.
With the help of the aquired knowledge, an effective
basis for decision-making can be created for
new plugin developers and the entry into plugin development
can be facilitated.

\end{english}
	

\tableofcontents		

%%%-----------------------------------------------------------------------------
\mainmatter                             % Hauptteil (ab hier arab. Seitenzahlen)
%%%-----------------------------------------------------------------------------

\chapter{Einleitung}
\label{cha:Einleitung}


\section{Motivation}
\label{sec:Motivation}

SoftwareentwicklerInnen arbeiten täglich mit verschiedensten 
Werkzeugen und Entwicklungsumgebungen, sogenannten IDEs 
(=Integrated Development Environment). Diese Plattformen 
bieten teils sehr unterschiedliche Funktionalitäten, die 
die Softwareentwicklung erleichtern sollen. Dabei bieten 
sie Unterstützung für verschiedenste Programmiersprachen 
und Technologien und binden zahlreiche Werkzeuge für 
spezifische Anwendungsfälle ein. Aufgrund des immer rascher
werdenden Entstehens von neuen Technologien bieten mehr 
und mehr IDEs Möglichkeiten zur Entwicklung von eigenen 
Plugins, welche dann auch an andere EntwicklerInnen 
bereitgestellt werden können. So können in kürzester 
Zeit neue Technologien unterstützt werden und 
EntwicklerInnen haben selbst die Macht darüber zu 
entscheiden welche Plugins sie nutzen möchten und 
welche nicht.

Vor der Entwicklung solcher Plugins ist es wichtig 
zu entscheiden für welche IDE das Plugin erstellt 
werden soll. Dabei spielen Aspekte wie zum Beispiel 
die Einfachheit und Flexibilität in der Entwicklung, 
der Umfang an angebotener Funktionalität, die Möglichkeit 
die Nutzerinteraktion und somit die User Experience 
zu steuern und viele weitere eine Rolle. 
Diese Bachelorarbeit versucht in diesen Bereichen 
einen Überblick zu schaffen und vergleicht hierfür 
die Plugin Entwicklung in zwei der momentan beliebtesten 
IDEs, Visual Studio Code und IntelliJ IDEA. Durch den 
Vergleich der beiden Produkte und dem Herausarbeiten 
und Aufbereiten der Unterschiede wird es anderen 
EntwicklerInnen erleichtert diese Entscheidung zu treffen.%
\cite{ArnoldKen1996TJpl,HagosTed2022BII:,KurbatovaZarina2021TIPa,StackOverflowSurvey2023}


\section{Ziel}
\label{sec:Ziel}

\section{Aufbau}
\label{sec:Aufbau}
\chapter{Grundlagen der Plugin Entwicklung}
\label{cha:Grundlagen}

\section{Entwicklungsumgebungen}
\label{sec:Entwicklungsumgebungen}

\subsection{Visual Studio Code}

Die erste offizielle Version von Visual Studio Code, häufig 
abgekürzt auch als VS Code, wurde im April 2016 \cite{VSCodeReleaseDate}
von Microsoft veröffentlicht. Die Idee hinter VS Code
war einen möglichst einfachen Code Editor anzubieten, 
welcher nur die wichtigsten und besten Funktionen für EntwicklerInnen 
beinhaltete. Es hob sich somit von anderen IDEs wie der Visual Studio 
Reihe von Microsoft ab, da es ein sehr leichtgewichtiger Editor war, 
welcher trotzdem mit einer großen Menge an Programmiersprachen arbeiten 
konnte und für diese auch Microsofts code completion namens „IntelliSense“ 
unterstützte. Weiters war Visual Studio Code das erste Produkt der Visual 
Studio Familie welches Cross-Plattform für Windows, Linux und OSX 
angeboten wurde \cite{VSCodePreview}.

Aus den Stack Overflow developer surveys der vergangenen Jahre kann 
der rasche Aufstieg von VS Code beobachtet werden. Während es im 
Jahr der Veröffentlichung nur von etwa 7,2 Prozent der EntwicklerInnen 
genutzt wurde, war es zwei Jahre später bereits (wenn auch knapp) 
das meistgenutzte IDE mit 34,9 \%. In der aktuellsten Umfrage von 
2023 war es der klare Sieger und wurde vom 73,71\% der Abstimmenden 
aktiv genutzt\cite{StackOverflowSurvey,StackOverflowSurvey2023}.

Ein Grund für diesen Erfolg mag vermutlich die Möglichkeit 
zur Entwicklung und zum Anbieten von Plugins sein. Durch die direkte 
Einbindung es Visual Studio Marketplace in VS Code bildete sich über die 
Jahre eine große Community die eine enorme Anzahl von Plugins 
entwickelt, verbessert und betreut. Durch solche, meist 
community-erstellte, Plugins kann VS Code auch eine enorme Anzahl von
Programmiersprachen unterstützen.


\subsection{IntelliJ IDEA}

IntelliJ IDEA wurde erstmals im Januar 2001 \cite{IntelliJIDEAWikipedia,IntelliJReleasePage}
von dem Unternehmen 
JetBrains veröffentlicht. Im Gegensatz zu Visual Studio Code handelt 
es sich bei IntelliJ um ein vollausgetattetes
\enquote{Integrate Development Environment}
(IDE) welches speziell auf die Entwicklung 
von Programmen in den Programmiersprachen Java, Kotlin und Groovy ausgelegt ist. 
IntelliJ IDEA wird in einer frei zu verwendenden, open source 
\enquote{Community Edition}, sowie in einer kommerziellen Form als 
\enquote{IntelliJ IDEA Ultimate} angeboten \cite{HagosTed2022BII:}. 

Aufgrund der Spezialisierung auf JVM kompatible Sprachen unterstützt 
die IntelliJ Community Edition nur eine relativ kleine Auswahl an 
Sprachen, Frameworks und Build Tools. Während IntelliJ IDEA Ultimate 
den Umfang an Features schon deutlich erweitert, bietet JetBrains auch 
noch weitere (kommerzielle) IDEs an. Diese sind alle für unterschiedliche 
Programmiersprachen oder Sprachfamilien ausgelegt. Einige der bekanntesten 
sind dabei CLion für die Sprachen C und C++, Rider für die .NET Sprachen, 
PhpStorm für PHP, WebStorm für JavaScript und viele weitere. Zum aktuellen 
Zeitpunkt sind es insgesamt elf verschiedene IDEs die von JetBrains 
angeboten werden und die alle auf der IntelliJ Platform basieren. Das 
bedeutet nicht nur, dass sich all diese IDEs in der Verwendung und im 
Aussehen sehr ähnlich sind, sondern auch, dass ein Plugin, welches für 
die allgemeine IntelliJ Platform entworfen wurde, relativ problemlos 
auch für mehrere IDEs dieser Form veröffentlicht werden kann \cite{IntelliJSDKDocumentation}.

Im Gegensatz zu Visual Studio Code ist IntelliJ ein eher schwergewichtiger
Editor, der sehr viel Funktionalität schon von Grund auf eingebaut 
hat. Die EntwicklerInnen sind hier nicht so stark auf Plugins angewiesen.
Dies lässt sich auch durch die Anzahl von Plugins erkennen, die auf dem 
JetBrains Marketplace angeboten werden. Für die IntelliJ Platform gibt 
es aktuell etwas über 7500 Plugins die in die IDE integriert werden können 
\cite{IntelliJMarketplace}.
Für Visual Studio Code sind es hingegen inzwischen über 51000 Plugins
\cite{VSCodeMarketplace}.


\section{Programmiersprachen}
\label{sec:Programmiersprachen}

\subsection{TypeScript}

Die TypeScript Programmiersprache wurde erstmalig am 1. Oktober 2012 
\cite{TypeScriptCodePlexArchived} von 
Microsoft in Form eines open-source Projekts veröffentlicht. Designed wurde sie 
von Anders Hejlsberg, der auch an der Entwicklung von C\# beteiligt war. 

Die grundsätzliche Idee der Sprache ist, eine typsichere, kompilierte, und somit 
bessere Version von JavaScript zu sein. JavaScript ist aufgrund des Erfolgszugs
des Internets zu einer sehr wichtigen Sprache geworden und war auch schon 2012 
aus den TOP Listen für Programmiersprachen nicht mehr wegzudenken 
\cite{StackOverflowSurvey,TIOBEIndex,PYPL}. 
Webseiten setzen heute sehr stark auf JavaScript, um durch interaktive Elemente 
die User Experience zu verbessern oder um neue Funktionalität anbieten zu können. 
Durch das Node.js runtime environment kann JavaScript nicht mehr nur im Browser 
verwendet werden, sondern es können auch Desktop, Server oder Mobile Anwendungen 
in JavaScript entwickelt werden. Durch diesen großen Umfang an Möglichkeiten die 
JavaScript dadurch bietet werden natürlich auch immer größere Projekte damit entwickelt. 
Und hier kommen die großen Schwächen von JavaScript immer mehr zu tragen. 
Je größer die Projekte werden und je mehr EntwicklerInnen an einem Projekt 
mitarbeiten, desto mehr Fehler entstehen aufgrund der fehlenden Typsicherheit
und des fehlenden Compilerschrittes. Diese Schwachstellen versucht TypeScript 
nun auszubessern.

TypeScript code wird mithilfe des TypeScript Compilers „tsc“ in einfache JavaScript 
Dateien transpiliert. Dadurch kann auf die Popularität und Verbreitung von JavaScript 
aufgebaut werden und TypeScript ist überall dort verwendbar, wo JavaScript 
ausführbar ist. Weiters ist TypeScript ist ein Superset von JavaScript. 
Es gilt also: „Any valid .js file can be renamed .ts and be compiled with other 
TypeScript file.” \cite{MaharryDanTR}. 

Jedoch bietet TypeScript eine Menge von Vorteilen 
gegenüber ihrer Basissprache.
\begin{itemize}
  \item Durch den Kompilierschritt mit dem tsc Compiler wird der 
    Code vor der Ausführung automatisch auf Validität geprüft. 
    Es entfällt also die Notwendigkeit für einen zusätzliches Linting 
    Tool wie JSLint. Dieser Compile-Schritt kann natürlich auch in eine 
    CI/CD Pipeline eingebunden werden, um auch bei Merges Feedback über die 
    Validität des Codes zu erhalten.
  \item Durch die statische Typisierung können Missverständnisse über 
    die Verwendung von Variablen vermieden werden. Auch die Unterstützung 
    durch verschiedene IDEs, zum Beispiel mittels IntelliSense kann durch die 
    Typen verbessert werden. Dies ist nicht nur bei der Zusammenarbeit hilfreich, 
    sondern kann auch die Arbeit jeder einzelnen EntwicklerIn beschleunigen.
  \item In TypeScript können Klassen erstellt werden, deren Properties mit 
    Zugriffsmodifikatoren (private/public) versehen sind.
  \item TypeScript unterstützt Vererbung, Interfaces und generische Programmierung.
  \item In TypeScript können bereits bestehende JavaScript Bibliotheken 
    wiederverwendet werden. Weiters ist es möglich durch zusätzliche Dateien 
    Typinformationen zu den bestehenden Bibliotheken zu liefern.
\end{itemize}

\subsection{Java}

Die Entwicklung der Programmiersprache Java begann im Jahr 1991 und sie 
wurde von den James Gosling, Mike Sheridan und Patrick Naughton designed \cite{WinnieDoug2021EJfA}.
Java wurde erstmals im Jahr 1995 von Sun Microsystems veröffentlicht. 
Im Januar 2010 wurde Sun Microsystems dann von der Oracle Corporation übernommen, 
welche seitdem auch Java weiterentwickelt.

Das Design und vor Allem die Syntax der Sprache war stark von C und C++ inspiriert \cite{ArnoldKen1996TJpl}, 
um anderen Entwicklern einen leichten Umstieg auf das neue Java zu ermöglichen. 
Allerdings versuchte Java die teils sehr komplexen (wenn auch effektiven) 
Sprachfeatures von C++ etwas zu vereinfachen. Java sollte eine simple, objektorientierte 
und robuste Sprache werden. Die Funktionalität die Java zu dem großen Erfolg verhalf, 
den sie später hatte, war das 
\begin{quote}\begin{english}\enquote{write once, run anywhere}\end{english}\end{quote}
(WORA) Prinzip, wie Sharan und Davis beschreiben \cite{SharanKishori2022BJ1f}. Im Gegensatz 
zu den zuvor gängigen Programmiersprachen muss Java nämlich für bestimmte 
Hardwarearchitekturen kompiliert werden. Java Programme werden zu einer Art 
Zwischensprache, dem sogenannten Java Bytecode kompiliert. Dieser Bytecode
kann dann von einer Java Virtual Machine (JVM) ausgeführt werden. Diese JVM ist
im Grunde ein eigenständiges Programm welche mit dem Java Runtime Environment 
(JRE) mitgeliefert wird. Ein einmal kompiliertes Java Programm kann also auf 
allen Geräten ausgeführt werden, auf denen ein passendes JRE installiert ist. 
So ist es zum Beispiel auch möglich Java für die Entwicklung von Android nativen
Apps auf Mobilgeräten zu benutzen.

Ein weiterer Vorteil gegenüber älteren Sprachen wie C++ ist die
automatisierte Speicherverwaltung. Diese funktioniert mithilfe eines 
sogenannten „garbage collectors“ welcher nicht mehr benötigten Speicher
am Heap bereinigt und freigibt. Man kann also beliebig neue Objekte im Speicher
allokieren und muss sich nicht um die deallokierung der zuvor erstellten Objekte
kümmern. Auf diese Weise können häufige Programmierfehler wie Memory Leaks fast 
vollständig unterbunden werden.

Java unterstützt sowohl das objektorientierte, das prozedurale als auch das funktionale 
Programmierparadigma. Der Fokus liegt allerdings stark auf der Objektorientierung. 
Dabei bietet Java Funktionalitäten zur Abstraktion durch Verwendung von Klassen, Information Hiding
mithilfe von Zugriffsmodifikatoren (public/private/protected/package), Vererbung, 
Interfaces, Polymorphismus, Überladen von Methoden, generischer Programmierung, 
Exception Handling und vieles mehr.

\section{Aufbau der Plugin API}
\label{sec:AufbauDerPluginAPI}

\subsection{Visual Studio Code}

Visual Studio Code bietet für Plugins zwei Arten der 
Interaktion, welche zusammenspielen um Plugins zu ermöglichen. 
Das Extension Manifest und die eigentliche API.
\subsubsection{Extension Manifest} 
  Das Extension Manifest befindet sich in der \enquote{package.json} 
  Datei. In dieser werden statische Einstellungen vorgenommen und
  Metainformationen über das Plugin bekannt gegeben. So kann hier unter
  anderem Name, Beschreibung, Herausgeber, Lizenzvereinbarungen und
  so weiter eingestellt werden. Weiters definiert das Manifest eine sogenannte
  \enquote{main} JavaScript oder TypeScript Datei und dazu passende
  \enquote{Activation Events} und \enquote{Contribution Points}.
  \begin{description}
    \item[Activation Events] bestimmen den Zeitpunkt an dem das Plugin zum ersten Mal
      aktiviert wird. Dabei wird die \enquote{activate} Funktion der zuvor definierten
      main Datei ausgeführt. Der Aktivierungszeitpunkt sollte immer so spät wie
      möglich gewählt werden, um VS Code möglichst wenig zu verlangsamen und
      das Plugin erst on demand zu Laden. Allerdings
      muss die Aktivierung natürlich passieren bevor die erste Funktionalität des
      Plugins erwartet wird. Typische Aktivierungsevents sind zum Beispiel \enquote{onCommand}
      , \enquote{onDebug}, \enquote{onView} oder \enquote{onStartupFinished}.
      Wurde das Plugin einmal aktiv, bleibt es auch aktiv bis VS Code wieder geschlossen
      wird oder das Plugin entfernt oder deaktiviert wird. Hierfür gibt es optional
      noch eine \enquote{deactivate} Funktion in der main Datei, welche für etwaige
      Aufräumarbeiten genutzt werden kann.
    \item[Contribution Points] legen fest welche Funktionalität das Plugin anbietet
      und somit auch welche zusätzlichen Elemente dem Nutzer in VS Code angezeigt werden sollen. % //TODO add some \ref and \pageref
      Hier ist es beispielsweise möglich Visual Studio Code mit neuen Befehlen (\enquote{Commands}),
      Menüs und Submenüs, Views für das Anzeigen von Plugindefiniertem Content,
      Keyboard Shortcuts, Unterstützung für neue Sprache, und vieles mehr auszustatten.
  \end{description}
\subsubsection{Visual Studio Code API} 
  Die eigentliche VS Code API kann im TypeScript Code (sowohl in der main, 
  als auch in anderen Dateien) genutzt werden. Hierfür wird einfach das \enquote{vscode}
  Modul importiert. Dieses beinhaltet eine vollständige definition der angebotenen
  Schnittstelle, auf welche programmatisch zugegriffen werden kann.
  \begin{JsCode}
    import * as vscode from 'vscode';

    export function activate(context: vscode.ExtensionContext) {
      vscode.window.showInformationMessage('Hello World!');
    }
  \end{JsCode}
  Über diese API kann dann zum Beispiel festgelegt werden, durch welchen Code
  die zuvor definierten Contribution Points implementiert werden sollen.
  Der Plugin Code wird in Visual Studio Code nicht im selben Prozess wie das
  Hauptprogramm ausgeführt, sondern abgekapselt in einem seperaten 
  \enquote{extension host process}. Dadurch kann verhindert werden, dass
  Plugins die Performance und die Interaktivität von VS Code negativ beinflussen 
  \cite{VSCodeArchitecture,VSCodeApproachToExtensibility}.
\subsubsection{Ablauf}
  Visual Studio Code analysiert zuerst das Extension Manifest des Plugins.
  Je nachdem welche Activation Points definiert sind, wird zu einem
  bestimmten Zeipunkt die activate Funktion aufgerufen. In dieser 
  können dann mithilfe der API Event Handler registriert werden. 
  Die registrierten Handler werden dann während der Ausführung und Verwendung
  von Visual Studio Code aufgerufen und können so beliebigen Code ausführen.
  Siehe Abbildung \ref{fig:diagram_VSCodeExtensionArchitecture}.
  \begin{figure}
    \centering
    \fbox{\includegraphics[width=.95\textwidth]{diagram_VSCodeExtensionArchitecture}}
    \caption{Übersicht über den Ablauf eines VS Code Plugins.}
    \label{fig:diagram_VSCodeExtensionArchitecture}
  \end{figure}   

\subsection{IntelliJ IDEA}

Der Aufbau der Plugin Architektur wirkt bei IntelliJ im ersten Moment genau
gleich wie bei Visual Studio Code. Es gibt nämlich auch hier gibt
es ein Plugin Configuration File, sowie ein Modul mit API Schnittstellen.
Der große Unterschied liegt allerdings in der Funktionsweise und der Interaktion
mit den Plugins und der Art wie der auszuführende Code angegeben wird.
\subsubsection{Plugin Configuration File}
  Die Konfiguration eines Plugins liegt in der \enquote{plugin.xml} Datei und
  beinhaltet, equivalent zum Extension Manifest in VS Code, alle für das Plugin
  notwendigen Meta-Informationen. So können auch hier Werte wie der Name,
  eine Beschreibung, die aktuelle Versionsnummer und so weiter angegeben werden.
  Für die Funktionen die das Plugin mitbringt gibt es Actions, Extension Points
  und Listener. Hier ist anzumerken, dass es sich sowohl bei den Extension Points,
  als auch den Listenern, immer direkt um eine Zuordnung eines Interfaces
  (meist definiert von IntelliJ) zu einer Implementierung (definiert durch das Plugin)
  handelt. Weiters ist es nicht nötig einen speziellen Aktivierungszeitpunkt
  festzulegen, da die zuordnung der auszuführenden Klassen sowieso durch 
  die Konfigurationsdatei festgelegt wird. Eine Besonderheit an IntelliJ ist,
  dass Plugins auch eigene Extension Points definieren können, um weiteren Plugins
  das erweitern des ursprünglichen Plugins zu erlauben.
\subsubsection{IntelliJ Platform SDK}
  Die API für IntelliJ Plugins ist in mehreren Paketen des IntelliJ 
  Platform SDK enthalten. Diese API enthält auch die unterschiedlichen
  Interfaces, welche dann in Form von Extension Points oder Listenern implementiert
  werden können. Die Implementierung eines Plugins kann in den Sprachen Java
  und Kotlin erledigt werden. Da die Plugin API allerdings auf Java basiert, können
  nicht alle Sprachfeatures von Kotlin problemlos genutzt werden.
\subsubsection{Ablauf}
  IntelliJ analysiert zuerst das Plugin Configuration File.
  Je nachdem welche Funktionalität vom Plugin angeboten wird, werden
  von IntelliJ automatisch die entsprechenden Event Handler 
  auf die unterschiedlichen Extension Points registriert. 
  Die registrierten Handler werden dann während der Ausführung und Verwendung
  von IntelliJ aufgerufen und können so beliebigen Code ausführen.
  Siehe Abbildung \ref{fig:diagram_IntelliJExtensionArchitecture}.
  \begin{figure}
    \centering
    \fbox{\includegraphics[width=.95\textwidth]{IntelliJExtensionArchitecture}}
    \caption{Übersicht über den Ablauf eines IntelliJ.}
    \label{fig:diagram_IntelliJExtensionArchitecture}
  \end{figure}   

\section{Funktionalität der Plugin API}
\label{sec:FunktionalitätDerPluginAPI}

\subsection{Visual Studio Code}

Die VS Code API erlaubt es VS Code durch Commands, Code Completion und Spracherweiterung, 
Themes, Custom Editor or Notebooks, Views, Source Control, Debugger, Tests und vielem mehr zu
erweitern. Um dies zu ermöglichen werden unter anderem auch Einstellungen, Datenspeicherung
und verschiedene Arten der Ein und Ausgabe von Daten bereitgestellt. In den folgenden Abschnitten
werden die wichtigsten Elemente genauer vorgestellt.

\subsubsection{Commands und Menüs}
  Commands ermöglichen es dem Plugin bestimmten Code sozusagen \enquote{auf Befehl} auszuführen.
  So können häufig wiederkehrende Aufgaben der User ganz einfach automatisiert werden.
  Um einen Command anzulegen, muss dieser im Extension Manifest definiert werden. Dabei muss
  das Plugin mindestens eine eindeutige Bezeichnung und zur Darstellung verwendeten Titel angeben.
  Optional können auch eine Kategorie, ein Icon, eine Kurzbezeichnung und eine Bedingung, 
  zu der der Command verwendbar wird, bestimmt werden.
  \begin{JsCode}
    "commands": [
      {
        "command": "vscodeplugindemo.helloWorld",
        "title": "Hello World",
      }
    ]
  \end{JsCode}

  Welcher Code dann ausgeführt wird, muss über die API festgelegt werden. So wird meist
  in der activate Funktion mithilfe von \enquote{registerCommand} oder \enquote{registerTextEditorCommand}
  ein Callback festgelegt welches aufgerufen werden soll. Wichtig ist hier, dass
  die Register Funktionen ein Disposable Objekt retournieren welches der API bekannt gegeben werden
  muss. Die API kümmert sich dann auch um das deaktivieren des Commands, falls zum Beispiel
  die Erweiterung deaktiviert werden sollte.
  \begin{JsCode}
    context.subscriptions.push(vscode.commands.registerCommand('vscodeplugindemo.helloWorld', () => {
      vscode.window.showInformationMessage('Hello World from VsCodePluginDemo!');
    }));
  \end{JsCode}

  Um einen Command aufzurufen können die Nutzer direkt nach dem Command suchen (Tastenkombination Strg+Shift+P).
  Komfortabler ist es allerdings den Nutzern direkt einen passenden Menüeintrag oder ein Keybinding
  bereitzustellen. Menüeinträge können dabei an verschiedenen Stellen im IDE eingehängt werden.
  Gängig sind hierfür die Titelleiste des Editors, verschiedene Kontext (Rechtsklick) Menüs, der Dialog
  zum Anlegen einer neuen Datei, die Titelleiste einer bestimmten View oder ein neues Submenü in der Menüleiste.
  Sowohl Menüs als auch Keybindings können im Manifest registriert werden.
  % // TODO Call commands from code
\subsubsection{Spracherweiterungen}
  Ein wichtiger Teil der VS Code Plugin API sind Language Extensions. Visual Studio Code unterscheidet
  bei diesen Funktionen nach Highlighting, Language Features und Snippets.
  \begin{description}
    \item[Highlighting] 
      Das einfache Syntax Highlighting wird in VS Code durch eine TextMate Grammatik erledigt \cite{TextMateGrammar}.
      Diese Grammatik wird dabei nicht nur für das Highlighting genutzt, sondern sie ist auch für die \enquote{Tokenization}
      zuständig. Durch eine Code-Analyse anhand der gegebenen Grammatik wird also der Text in kleine 
      zusammengehörige Abschnitte (sogenannte Tokens) unterteilt. Diese Tokens werden dann zusätzlich noch 
      klassifiziert, sodass zum Beispiel zwischen Kommentaren, Strings, RegExen und Code unterschieden werden kann.
      Die Grammatik wird hierfür in einer einfachen .json Datei im Plugin Projektordner abgelegt und dann per Manifest
      unter dem Contribution Point \enquote{grammar} eigebunden. Mithilfe socher Grammatiken können auch bereits bestehende
      Grammatiken erweitert werden. Bei der Auswahl von Scopes, die durch die Grammatik definiert werden, sollte man sich
      an die Naming Conventions von TextMate halten, da diese vordefinierten Scopes von vielen Themes unterstützt werden.
      Um die kategorisierung von bestimmten Tokens noch genauer zu erledigen, gibt es auch \enquote{Semantic Highlighting} 
      in VS Code. Es kann in der API programmatisch ein \enquote{DocumentSemanticTokensProvider} registriert werden,
      welcher den Code analysiert und zusätzliche (und zum Beispiel kontextabhängige) Informationen über die Tokens bereitstellt.
    \item[Language Features] 
      Auch hier gibt es statisch definierte und programmatische Language Features.
      Statisch kann unter dem Contribution Point \enquote{languages} eine Reihe von Informationen angegeben werden,
      die es VS Code erlauben die User Experience stark zu verbessern. So kann man unter anderem Festlegen,
      mit welchen Zeichen Kommentare eingeleitet werden, oder welche Klammern es gibt, damit VS Code das Zuklappen erlauben kann.
      Zusammengehörige Paare von Zeichen (also Klammern, Anführungszeichen, u.s.w.) können automatisch geschlossen werden.
      Und sogar für das automatische Einrücken der nächsten Zeile bei einem Zeilenumbruch kann eine Regel erstellt werden.
      Programmatisch ist das erweitern von Language Features etwas komplexer, allerdings steigt natürlich auch die
      Menge an Möglichkeiten. In der API können verschiedene Provider registriert werden, durch die Features wie
      \enquote{Go to Definition}, \enquote{IntelliSense} code completion oder diagnostische Fehleranalysen und
      entsprechende Verbesserungsvorschläge ermöglicht werden. Grundsätzlich ist es auch möglich für einzelne
      Features einen Provider zu registrieren, allerdings empfiehlt es sich bei der Einbindung einer neuen
      Sprache einen \enquote{Language Server} und das Language Server Protocol zu nutzen. Diese Lösung
      bringt nicht nur Performanceverbesserungen im Editor, sonder der Language Server kann auch für 
      andere Editoren wiederverwendet werden, ohne ihn komplett neu implementieren zu müssen.
      % //TODO results of existing lsp can be gotten by calling built-in commands. need some further research for this...
    \item[Snippets] 
      Snippets sind eine sehr einfache Form der Spracherweiterung. Es wird unter dem Contribution Point
      \enquote{snippets} einfach eine Datei mit den Vorlagen angegeben. Eine Vorlage enthält dabei immer
      ein Kürzel für welches das Snippet vorgeschlagen werden soll, den zu ersetzenden Text und optional
      eine Beschreibung. Dabei können im Text auch Platzhalter genutzt werden an die der Cursor beim Einsetzen
      des Snippets springt.
  \end{description}
\subsubsection{Benutzereingaben}
  Für die Eingabe von Daten bietet Visual Studio Code die Quick Pick API, die File Picker API
  und den Configuration Contribution Point.
  \begin{description}
    \item[Quick Pick API] gibt dem Plugin eine Möglichkeit den Nutzern ein einfaches Eingabefenster
      anzuzeigen. Dabei kann ein Fenster mit bereits vorgegebenen Auswahlmöglichkeiten durch
      den Aufruf von \enquote{showQuickPick} oder \enquote{createQuickPick} erstellt werden.
      Alternativ kann man mit den Funktionen \enquote{showInputBox} oder \enquote{createInputBox}
      die Nutzer auch selber einen Text eingeben lassen. Die show Funktionen bieten dabei immer
      eine einfache vorgefertigte Implementierung an. Falls diese Option nicht ausreichend ist,
      kann mit den create Funktionen auch eine komplexere Implementierung angegeben werden.
      Weiters kann auch eine Validierung des Inputs vorgenommen werden.
      Möchte man mehrere solcher Dialoge als Abfolge hintereinander anzeigen, so muss dies leider
      selber programmiert werden. Hierfür gibt es aber ein Beispiel im quickinput-sample
      im vscode-extension-sample repository \cite{VSCodeExtensionSamples}. 
    \item[File Picker API] erlaubt das Auswählen von Ordnern oder Dateien aus dem Dateisystem
      des Betriebssystems mit der Funktion \enquote{showOpenDialog}. Dabei können optionen angegeben
      werden die beeinflussen ob Ordner und/oder Dateien gewählt werden dürfen, ob mehrere Elemente
      selektiert werden dürfen, ob nach Dateinamen gefiltert werden soll und so weiter.
    \item[Configuration Contribution Point] ermöglichen das Festlegen von Einstellungen
      die von den Nutzern eingegeben und vom Plugin ausgelesen werden können. Hier können
      Einstellungen vom Typ number, string und boolean definiert werden, welche dann direkt
      im User Interface der VS Code Einstellungen bearbeitet werden können.
      Einfache object und array Properties können auch im UI dargestellt werden,
      allerdings dürfen diese keine verschachtelten Objekte oder Arrays enthalten.
      Ansonsten wird in den Einstellungen nur auf die manuell zu bearbeitende 
      \enquote{settings.json} Datei verwiesen.
      Für die Validierung der Einstellungen können Validierungsproperties
      von JSON Schema verwendet werden. Es ist also zum Beispiel möglich
      ein maximum/minimum, ein Regular Expression pattern oder ein enum Array
      mit erlaubten Werten angegeben werden.
      Zusätzlich ist es zu jeder Einstellung möglich einen Titel und eine Beschreibung
      anzugeben, wobei es sogar Beschreibungen gibt, welche Markdown formattierungen
      enthalten dürfen.
  \end{description}
\subsubsection{Ausgaben und Anzeigen}
  Um den Usern auch Feedback über die Ausführung des Plugin Codes zu geben, 
  ist in VS Code für drei allgemeine Anwendungsfälle vorgesorgt. 
  
  Um den Usern eine kurze 
  Rückmeldung zu geben können am besten Notifications genutzt werden. Diese zeigen eine Kurze 
  Nachricht an, welche im Stil einer Information, einer Warnung oder einer Error Meldung 
  dargestellt werden kann. Um einen längeren Fluss von Ausgaben (wie zum Beispiel Log-Nachrichten 
  des Plugins) anzuzeigen können Output Channels genutzt werden. An diese können Textzeilen nach 
  und nach angehängt werden und sie werden dem User dann in einem Terminalartigen Fenster präsentiert. 
  In vielen Fällen reicht es schon als Feedback einen einfachen Ladebalken anzuzeigen. So kann dem 
  User klar gemacht werden, dass das Plugin immer noch arbeitet und noch kein Fehler aufgetreten 
  ist. Für diesen Anwendungsfall kann die Progress API genutzt werden.

  Eine etwas komplexere Anzeige bieten Views, die die sogenannte Workbench erweitern.
  Mit der Tree View API kann eine einfach Baumstruktur, ähnlich der 
  Dateiübersicht in der Explorer View, dargestellt werden. Für diese Implementierung
  muss ein TreeDataProvider erstellt werden, welcher die Baumstruktur und
  den Inhalt vorgibt. Die Webview API bietet im Gegensatz dazu sehr viel mehr
  Optionen. Diese kann in einem View eine Art \enquote{iframe} anzeigen, in welchem
  dann HTML Inhalte dargestellt werden können. Dabei kann auch JavaScript und CSS Code eingebunden
  werden, es können Nachrichten vom Plugin an die Webview und zurück geschickt werden, 
  es können Kontextmenüs in die View eingebunden werden und der Zustand der View
  kann persistiert werden.

\subsection{IntelliJ IDEA}

Das IntelliJ Platform SDK enthält einen sehr großen Umfang von Features und Extension Points
die durch ein Plugin erweitert werden können. Einige wichtige Teile der API werden in den 
folgenden Abschnitten genauer beschrieben.

\subsubsection{Actions und Menüs}
  Actions in IntelliJ funktionieren fast ident zu den Commands aus VS Code. Es handelt sich um
  einen vom Plugin definierten Code-Block, welcher von den Usern zum Beispiel über Menüeinträge
  angestoßen werden kann. Eine Action ist dabei eine einfache Java Klasse, welche von \enquote{AnAction}
  abgeleitet wird. Dabei muss die Methode \enquote{actionPerformed} überschrieben werden. Diese 
  enthält den Code der von der Action ausgeführt wird. Optional kann (und sollte) auch die
  \enquote{update} Methode überschrieben werden, durch welche bestimmt wird wann die Action 
  aktiviert oder versteckt ist.

  Im Plugin Configuration File wird festgelegt wo und wie die programmierten Actions angezeigt
  werden. Dabei wird im Abschnitt \enquote{actions} ein \enquote{action} Element erstellt.
  Dieses hat für gewöhnlich eine eindeutige ID, eine Klasse mit der Code-Implementierung und
  einen Text, welcher zur Anzeige verwendet wird. Zusätzlich können eine Beschreibung und
  ein Icon festgelegt werden. Es können Gruppenzuordnungen bestimmt werden, die bestimmen
  wo und wie die Action angezeigt wird. Es können Keyboard Shortcuts bestimmt werden. Und
  es kann mit \enquote{override-text} ein alternativer Text angegeben werden, der nur an
  bestimmten Orten angezeigt wird.
  
  \begin{XmlCode}
    <actions>
        <action id="my.simple.DemoAction"
                class="my.simple.DemoAction" 
                text="Demo Action">
            <add-to-group group-id="ToolsMenu" anchor="first"/>
        </action>
    </actions>
  \end{XmlCode}

\subsubsection{Services}
  IntelliJ erlaubt es einem Plugin Services zu definieren, welche dann auf drei Ebenen
  in jeweils zwei Varianten implementiert werden können. Instanzen solcher Services können
  dann an beliebigen Stellen im Plugin Code verwendet werden.

  Die Ebene auf der ein Service erstellt wird, bestimmt wie viele Instanzen dieses 
  Services existieren können. Dabei gibt es das application-level, welches den Service
  als globales Singleton anbietet. Und project-level und module-level Services, bei welchen
  für jedes geöffnetet Projekt bzw. Modul je eine Instanz des Services besteht. Allerdings 
  wird empfohlen aus Effizienzgründen keine Services auf Modul-Level zu erstellen.
  In Bezug auf die Varianten gibt es Light Services und normale Services.

  Die Light Services sind dabei einfache Klassen, welche mit der Annotation \enquote{@Service}
  versehen sind. Light Services sind sehr effizient, allerdings gibt es einige Einschränkungen.
  So können beispielsweise keine anderen Services per Dependency Injektion injiziert werden.
  Gewöhnliche Services haben diese Einschränkungen nicht. Bei ihnen wird ein beliebiges Interface und
  eine dazugehörige Implementierung definiert. Diese werden daraufhin im Plugin Configuration File
  unter den Extension Points \enquote{applicationService} oder \enquote{projectService} registriert.
  Die Project-Level Services erhalten dabei sowohl als Light Service, als auch als normaler Service, 
  eine referenz auf das aktuelle Projekt.

\subsubsection{Listeners und Extension Points} % // TODO section could probably be removed

\subsubsection{Spracherweiterungen und PSI}


\subsubsection{User Interface Komponenten}
  Auch in IntelliJ gibt es einige vorgefertigte UI Komponenten. Das dient vor allem dazu, die
  User Experience über verschiedene Plugins hinweg möglichst einheitlich und im Stile
  der IntelliJ Platform zu halten. Häufig verwendete Komponenten sind hierbei Dialoge, Popups,
  Notifications und Tool Windows.
  \begin{description}
    \item[Dialoge] gibt dem 
    \item[Popups] erlaubt dada
    \item[Notifications] erlaubt dadu
    \item[Tool Windows] erlaubt
  \end{description}

\subsection{IntelliJ Flora Plugins}

In der Plugin Dokumentation von JetBrains wird zu Beginn 
empfohlen sich noch einmal gründlich zu überlegen, ob man 
für die von einem gewünschte Funktionalität wirklich ein 
vollwertiges Plugin benötigt. Häufig kommt es nämlich vor, 
dass nur bestimmte kleine Tasks innerhalb des IDEs 
automatisiert werden sollen \cite{IntelliJSDKDocumentation}. Hierfür schlägt JetBrains 
einige leichtgewichtige Alternativen vor. Eine nennenswerte 
Alternative ist das „Flora Plugin“ für das IntelliJ IDEA. 

Flora kann über die Einstellungen des IntelliJ IDEA 
im Abschnitt „Plugins“ installiert werden.

\begin{figure}
    \centering
    \fbox{\includegraphics[width=.95\textwidth]{flora_plugin}}
    \caption{Flora Plugin im IntelliJ Plugin Marketplace.}
    \label{fig:FloraPlugin}
\end{figure}    
 
Das Plugin sucht dann in den geöffneten Projektverzeichnissen
nach ausführbaren JavaScript oder Kotlin Script „micro plugin“ 
Dateien. Diese müssen sich in einem Ordner namens „.plugins“ 
befinden und auf „.plugin.js“ oder „.plugin.kts“ enden \cite{FloraPluginMarketplace}.
Innerhalb diese Plugin Dateien kann über die Variable „ide“ auf 
die angebotene Schnittstelle zugegriffen werden. Diese erlaubt 
es unter anderem Actions, Keyboard Shortcuts, Services und 
ToolWindows zu erstellen.

\begin{figure}
    \centering
    \fbox{\includegraphics[width=.75\textwidth]{flora_codeCompletion}}
    \caption{Übersicht über die API des Flora Plugins.}
    \label{fig:FloraPluginAPI}
\end{figure}    
 
Flora Plugins bieten sich vor allem dann an, wenn eine projektspezifische 
Aufgabe automatisiert werden soll. Hier sind vor Allem die 
Leichtgewichtigkeit der Plugins und die Schnelle, mit der ein 
einfaches Plugin entwickelt werden kann, von großem Vorteil. 
Weiters spricht für diesen Anwendungsfall, dass der Plugin Code 
direkt im Projektordner abgelegt wird und somit auch in einem Version 
Control System wie Git mit abgelegt werden kann.

% //TODO Frage: Sehr viele Absätze referenzieren "die Doku". soll jeder Absatz die entsprechende Sub-Page referenzieren?
% //            Oder gibt es eine Möglichkeit für ein ganzen Kapitel die Doku zu referenzieren o.ä.?
\chapter{Anforderungen an den Prototyp}
\label{cha:Prototyp}

\section{Aufbau}
\label{sec:Prototyp_Aufbau}


\include{chapters/entwicklungVsCode}
\chapter{Entwicklung des Prototpys für IntelliJ}
\label{cha:EntwicklungIntelliJ}

\section{Design}
\label{sec:EntwicklungIntelliJ_Design}

Wie schon in Kapitel \ref{cha:EntwicklungVsCode}, setzt sich das 
beschriebene Plugin in IntelliJ durch die Komponenten zusammen,
die in Abbildung \ref{fig:diagram_IntelliJDesign-Simplified} abgebildet sind.

\begin{figure}
    \centering
    \includegraphics[width=.95\textwidth]{diagram_IntelliJDesign-Simplified}
    \caption{Stark vereinfachte Übersicht über das Design des Plugins in IntelliJ.}
    \label{fig:diagram_IntelliJDesign-Simplified}
\end{figure}

Da die beiden Plugins im Aufbau sehr ähnlich sind, sind auch die Aufgaben der
Einzelnen Komponenten beinahe deckungsgleich. Die Unterschiede liegen eher in
den Details.

Der \emph{SimpleChangeDocumentListener} übernimmt hier die Aufgabe,
des Beobachten des geöffneten Dokuments auf Veränderungen. Wird
eine Änderung festgestellt, so wird diese an den \emph{RecentChangesService}
zur Speicherung übergeben. Im Gegensatz zum \emph{RecentChangeStorage}
ist dieser allerdings als expliziter Service deklariert, der von IntelliJ
selbst verwaltet wird.

Die Komponenten für Einstellungen teilen sich auf 
\emph{RecentChangesSettingsConfigurable} und \emph{RecentChangesSettingsService}
auf. Die Klasse \emph{RecentChangesSettingsConfigurable} kümmert sich dabei
um die Darstellung und die Interaktivität der Einstellungen in der Benutzerschnittstelle.
Die Klasse \emph{RecentChangesSettingsService} wird wieder als Service 
angeboten und kümmert sich um das Speichern, Auslesen und Persistieren
der Einstellungen.

Die Codevervollständigung wird im IntelliJ Plugin durch den
\emph{RecentChangesCompletionContributor} durchgeführt. Dieser
sucht im \emph{RecentChangesService} nach Änderungen, die auf
das ausgewählte Wort passen und gibt diese als Vorschläge zurück.

Für das Einsetzen der Änderungen im Editor wird die Klasse
\emph{ApplyRecentChange} verwendet, die als Action registriert ist.
Bei der Action ist wie bereits im VS Code Plugin eine Tastenkombination
zum schnelleren Aufrufen hinterlegt.
Sollte keine passende Änderung gefunden werden, so wird direkt im
Editor ein Warnhinweis mit einer entsprechenden Meldung angezeigt.

Die Klasse \emph{RecentChangesToolWindowFactory} ist für die Darstellung
einer UI-Komponente zuständig, die dem TreeView aus dem VS Code Plugin
ähnelt. Sie ließt dafür den Zustand des \emph{RecentChangesService} aus
und baut eine entsprechende Baumstruktur auf. Um die Ansicht auch zum richtigen
Zeitpunkt aktualsieren zu können, wird der Service mithilfe eines
Observer-Patterns \cite{2005Dp:e} beobachtet.

Die Komponente \emph{IsRecentChangesRunningService} existiert für den Fall,
das zwei Instanzen von IntelliJ gleichzeitig auf einem Gerät gestartet werden.
Beim Start der Anwendung, müssen nämlich einige Initialisierungen vorgenommen
werden, die nur einmalig durchgeführt werden dürfen. Gegen den
\emph{IsRecentChangesRunningService} kann auf diese Weise geprüft werden, ob
die Initialisierung noch nötig ist, oder bereits gemacht wurde.


\section{Implementierung}
\label{sec:EntwicklungIntelliJ_Implementierung}

\subsection{Aufsetzen des Projektes}

% //TODO source

Das Erstellen eines neuen Plugin-Projektes kann in IntelliJ über den 
\enquote{New Project Wizard} erledigt werden. Dafür muss einfach
im Programm IntelliJ IDEA das Menü \emph{File -> New -> Project...}
gewählt werden. Hier gibt auf der linken Seite des Fensters eine 
Liste mit Projekt-Vorlagen, in welcher sich auch der Eintrag 
\emph{IDE Plugin} befindet. Für diesen Projekttyp kann dann unter anderem
ein Name für das Projekt gewählt werden. Dieser Name wird initial auch als
Name des Plugins verwendet.

Die durch IntelliJ generierte Ordnerstruktur kann (ausschnittsweise) in 
Abbildung \ref{fig:intellij_generated_structure} betrachtet werden.
Relevant ist hier vor allem die Datei \emph{plugin.xml}, die das 
enstprechend vorbereitete Plugin Manifest beinhaltet. Der Ordner
\emph{kotlin} ist als Verzeichnis für den Plugin-Code vorgesehen.
Er kann allerdings je nach Präferenz auch durch einen Ordner \emph{java}
ersetzt werden. Ein Ordner für Tests wird nicht automatisch generiert
und muss manuell hinzugefügt werden.

\begin{figure}
    \centering
    \includegraphics[width=.35\textwidth]{intellij_generated_structure}
    \caption{Ausschnitt der durch \emph{IntelliJ IDEA} generierten Ordnerstruktur.}
    \label{fig:intellij_generated_structure}
\end{figure}   

\subsection{Entwicklung}

\subsubsection{RecentChangesService}

Die Klasse \emph{RecentChangesService} (dargestellt in
Abbildung \ref{fig:diagram_IntelliJDesign-Detail_Service}) 
wird in der Form eines
Services auf Applikationsebene implementiert. Dies wird 
über die Annotation \emph{@Service(Service.Level.APP)} festgelegt. 
Auf diese Weise kann eine Instanz der Klasse in der gesamten Anwendung
bereitgestellt werden. Die statische Methode \emph{getInstance}
abstrahiert dabei die Aufrufe der IntelliJ API, die nötig sind um eine
Referenz auf diese Instanz zu erhalten. Genau wie im VS Code Plugin,
werden die Änderungen in einer \emph{EvictingQueue} von 
\emph{SimpleDiff}-Objekten gespeichert. Allerdings muss diese
Warteschlange hier nicht selbst implementiert werden, da es in
der Bibliothek \emph{Guava} von Google bereits eine etablierte 
Implementierung gibt. % //TODO source
Diese Bibliothek kann einfach
in der Datei \emph{build.gradle.kts} eingebunden werden.
Die verschiedenen Methoden der Klasse \emph{RecentChangesService} erlauben
das Einfügen, sowie das Auslesen, von \emph{SimpleDiff}-Objekten aus
der darunterliegenden Datenstruktur. Über die Methoden \emph{addChangeListener},
\emph{removeChangeListener} und \emph{notifyListeners} wird ein Observer-Pattern
abgebildet. Observer, welche die eigens erstellte Schnittstelle 
\emph{RecentDiffsChangedListener} implementieren, können sich also
beim \emph{RecentChangesService} anmelden. Sie werden dann bei Änderungen 
an den Daten über die Methode \emph{notifyChanged} notifiziert.
Da es sich bei der Klasse \emph{RecentChangesService} aufgrund der Implementierung
als IntelliJ Service um eine Singleton-Klasse % //TODO source
handelt, wird zusätzlich eine Methode \emph{reset} benötigt, um beim Testen
die Unabhängigkeit der verschiedenen Unit-Tests zu erhalten. Da für
die einzelnen Testfälle keine neuen Instanzen erzeugt werden können, muss
die Klasse also vor jedem Test zurückgesetzt werden.

\begin{figure}
    \centering
    \includegraphics[width=.95\textwidth]{diagram_IntelliJDesign-Detail_Service}
    \caption{Detaillierte Darstellung des \emph{RecentChangesService}.}
    \label{fig:diagram_IntelliJDesign-Detail_Service}
\end{figure}

\subsubsection{SimpleChangeDocumentListener}

Der detaillierte Aufbau der Komponente \emph{SimpleChangeDocumentListener}
kann in Abbildung \ref{fig:diagram_IntelliJDesign-Detail_Listener} betrachtet werden.
Eine Instanz der Klasse wird beim ersten Start von IntelliJ als so registriert, 
dass er über die Änderungen in \emph{allen} geöffneten Dokumenten informiert wird. 
Hierfür muss die Schnittstelle \emph{DocumentListener} implementiert werden.
Diese definiert unter anderem die Methodensignaturen \emph{beforeDocumentChange}
und \emph{documentChanged}. Durch das Zusammenspiel dieser beiden Methoden
wird der gewünschte Debounce-Effekt erzeugt. Zu Beginn einer Änderung
wird in der Methode \emph{beforeDocumentChange} der aktuelle Text aus der 
unveränderten Datei ausgelesen. In der Methode \emph{documentChanged} wird
dann ein Timer gestartet (oder neu gestartet, falls er bereits laufen sollte).
Erst nach Ablauf des Timers (ohne einer weiteren Eingabe) wird
eine Änderung als abgeschlossen erkannt. Sobald dies geschieht, wird 
der Algorithmus \emph{diff-match-patch} verwendet um die Änderung zu analysieren.
Wird daraufhin festgestellt, dass es sich um eine einfache Änderung handelt,
so wird diese in den Speicher des \emph{RecentChangesService} eingefügt.
Zu beachten ist in dieser Klasse weiters die private Methode 
\emph{getOriginalTextFromDocument}. Dieser komplexe Aufruf
ist in IntelliJ notwendig, um den reinen Text des veränderten Dokuments
zu erhalten. Die Klasse \emph{Document} hätte zwar eigentlich
eine Methode \emph{getText}, dieser Text befindet sich aber möglicherweise
in einem Zwischenzustand, in dem IntelliJ spezielle Zeichenketten 
einsetzt, um die Anzeige der Codevervollständigung zu erleichtern.
% //TODO source IntelliJIdeaRulezzz https://intellij-support.jetbrains.com/hc/en-us/community/posts/206752355-The-dreaded-IntellijIdeaRulezzz-string
Um die originale Datei (und nicht eine modifizierte Kopie) zu erhalten
müssen also einige Umwege gegangen werden.

\begin{figure}
    \centering
    \includegraphics[width=.95\textwidth]{diagram_IntelliJDesign-Detail_Listener}
    \caption{Detaillierte Darstellung des \emph{SimpleChangeDocumentListener}.}
    \label{fig:diagram_IntelliJDesign-Detail_Listener}
\end{figure}

\subsubsection{ApplyRecentChange}

Text

\begin{figure}
    \centering
    \includegraphics[width=.95\textwidth]{diagram_IntelliJDesign-Detail_Action}
    \caption{Detaillierte Darstellung der Action \emph{ApplyRecentChange}.}
    \label{fig:diagram_IntelliJDesign-Detail_Action}
\end{figure}

\subsubsection{RecentChangesCompletionContributor}

Text

\begin{figure}
    \centering
    \includegraphics[width=.95\textwidth]{diagram_IntelliJDesign-Detail_Contributor}
    \caption{Detaillierte Darstellung des \emph{RecentChangesCompletionContributor}.}
    \label{fig:diagram_IntelliJDesign-Detail_Contributor}
\end{figure}

\subsubsection{RecentChangesToolWindowFactory}

Text

\begin{figure}
    \centering
    \includegraphics[width=.95\textwidth]{diagram_IntelliJDesign-Detail_ToolWindow}
    \caption{Detaillierte Darstellung der \emph{RecentChangesToolWindowFactory}.}
    \label{fig:diagram_IntelliJDesign-Detail_ToolWindow}
\end{figure}

\subsubsection{Einstellungen}

Text

\begin{figure}
    \centering
    \includegraphics[width=.95\textwidth]{diagram_IntelliJDesign-Detail_Settings}
    \caption{Detaillierte Darstellung der \emph{Settings} Komponenten.}
    \label{fig:diagram_IntelliJDesign-Detail_Settings}
\end{figure}

\section{Tests}
\label{sec:EntwicklungIntelliJ_Tests}

\section{Publishing}
\label{sec:EntwicklungIntelliJ_Publishing}

\section{CI/CD}
\label{sec:EntwicklungIntelliJ_CICD}
\chapter{Bewertungskriterien}
\label{cha:Kriterien}

\section{Popularität der Entwicklungsumgebung}
\label{sec:Kriterien_Popularität}

\subsection{Visual Studio Code}

\subsection{IntelliJ IDEA}


\section{Performance}
\label{sec:Kriterien_Performance}

\subsection{Visual Studio Code}

\subsection{IntelliJ IDEA}


\section{Feature Umfang}
\label{sec:Kriterien_FeatureUmfang}

\subsection{Visual Studio Code}

\subsection{IntelliJ IDEA}


\section{Intuitivität der API}
\label{sec:Kriterien_Intuitivität}

\subsection{Visual Studio Code}

\subsection{IntelliJ IDEA}


\section{Dokumentation der API}
\label{sec:Kriterien_Dokumentation}

\subsection{Visual Studio Code}

\subsection{IntelliJ IDEA}


\section{Testbarkeit des Plugins}
\label{sec:Kriterien_Testbarkeit}

\subsection{Visual Studio Code}

\subsection{IntelliJ IDEA}


\section{Möglichkeiten des Publishings}
\label{sec:Kriterien_Publishing}

\subsection{Visual Studio Code}

\subsection{IntelliJ IDEA}


\section{Installationsprozess des Plugins}
\label{sec:Kriterien_Installationsprozess}

\subsection{Visual Studio Code}

\subsection{IntelliJ IDEA}

\chapter{Vergleich anhand der Bewertungskriterien}
\label{cha:Vergleich}

\section{Popularität der Entwicklungsumgebung}
\label{sec:Vergleich_Popularität}

\subsection{Visual Studio Code}

Laut der Stack Overflow Developer Survey von 2023 ist
VS Code der große Spitzenreiter der IDEs. Es wurde von
73,71\% der EntwicklerInnen angegeben,
dass sie VS Code verwenden \cite{StackOverflowSurvey2023}.
Auch der PYPL Index bestätigt die Beliebtheit von VS Code~\cite{PYPL}.
Dieser Index misst die Anzahl von Suchanfragen in der Google Suchmaschine
und reiht VS Code auf den zweiten Platz hinter Visual Studio.

Die Anzahl von Extensions, die im Visual Studio Marketplace angeboten werden,
beträgt aktuell über 54.000 \cite{VSCodeMarketplace}. Allerdings sind 
darunter auch viele Extensions, die kaum verwendet werden und die nur
wenige Downloads vorweisen können. Zählt man nur die 
Extensions, die über eine Million Downloads erreicht haben,
kommt man auf etwas über 350 Extensions.

\subsection{IntelliJ IDEA}

In der Stack Overflow Developer Survey gaben 2023
26,42\% der EntwicklerInnen an, IntelliJ IDEA zu verwenden,
womit es in der Umfrage den dritten Platz erreichte~\cite{StackOverflowSurvey2023}.
PYPL stuft IntelliJ auf Platz sieben ihres Index ein \cite{PYPL}.
Allerdings darf hierbei nicht vergessen werden, dass Plugins
für die IntelliJ Platform auch in anderen JetBrains IDEs installiert
werden können. Berechnet man aus den Daten der Stack Overflow Umfrage
eine gemeinsame Beliebtheit der IntelliJ Platform IDEs, so findet man,
dass 52,87\% der EntwicklerInnen eine solche Programmierumgebung nutzen \cite{StackOverflowSurvey}.
Weiters muss beachtet werden, dass die Zielgruppe von IntelliJ IDEA
vor allem Java EntwicklerInnen sind. Bei Umfragen mit Java-EntwicklerInnen 
schneidet IntelliJ IDEA meist mit dem ersten Platz ab \cite{JRebelIDEs,JRebelDeveloperProductivityReport,BetterprojectsfasterPouplarityIndex}.

Im JetBrains Marketplace werden für die gesamte IntelliJ Platform aktuell
etwas über 7.800 Plugins zum Download angeboten. Zählt man nur die 
Plugins mit mehr als einer Million Downloads, kommt man auf
knapp über 100 Plugins.

\subsection{Vergleich}

Während es sich bei VS Code um einen allgemeinen Code-Editor handelt,
der sehr vielseitig eingesetzt werden kann, bietet die IntelliJ-Plattform
eher spezifische Werkzeuge, die in erster Linie auf die Entwicklung in
einer einzelnen Programmiersprache ausgerichtet sind.
Obwohl VS Code zwar insgesamt der häufiger genutzte Editor ist,
sind die JetBrains IDEs für das Programmieren in einer 
speziellen Sprache (zum Beispiel IntelliJ IDEA für Java) oft die
beliebtere Wahl.
    

\section{Ausführungszeit und Hardwareanforderungen}
\label{sec:Vergleich_Performance}

\subsection{Visual Studio Code}

Mithilfe der Anwendung \emph{AppTimer} von PassMark wurde
die durchschnittliche Startzeit von VS Code ermittelt \cite{PassMarkAppTimer}.
AppTimer wurde so konfiguriert, dass VS Code automatisch gestartet
und wieder geschlossen wird. Beim Start wurde von VS Code ein
einfacher Projektordner geladen. Der AppTimer maß bei jedem Start
die Zeit die VS Code brauchte, um in einen Zustand zu gelangen,
in dem Nutzereingaben angenommen werden konnten.
Diese Messung wurde in 100 Iterationen wiederholt.
Daraus ergab sich eine durchschnittliche Startzeit von 0.294 Sekunden.

Die Hardwareanforderungen sind gering. VS Code benötigt weniger
als 500 MB Festplattenspeicher, 1 GB RAM und eine Prozessorgeschwindigkeit
von 1.6 GHz, um lauffähig zu sein.

\subsection{IntelliJ IDEA}

Die Messungen mithilfe der AppTimer-Anwendung wurden auch
für IntelliJ IDEA Ultimate in 100 Iterationen wiederholt.
Hier war die durchschnittliche Startzeit bei 8.123 Sekunden.

Für die Hardwareanforderungen gibt IntelliJ IDEA
mindestens 3.5 GB Festplattenspeicher, 2 GB RAM und eine
moderne CPU vor.

\subsection{Vergleich}

Da es sich bei VS Code nur um einen leichtgewichtigen Editor handelt,
hat dieser auch einen klaren Vorteil bei der Performance 
und den Hardwareanforderungen.
IntelliJ IDEA ist eine vollständige IDE und hat daher auch längere
Startzeiten und höhere Hardwareanforderungen.
Auf modernen Geräten sollten beide Entwicklungsumgebungen
problemlos ausführbar und verwendbar sein. Allerdings kann VS Code hier
einen Vorteil bieten, wenn gleichzeitig andere, ressourcenintensive Programme
auf dem Rechner laufen, oder Hardwarelimitierungen vorliegen.


\section{Feature-Umfang}
\label{sec:Vergleich_FeatureUmfang}

\subsection{Visual Studio Code}

VS Code unterstützt viele grundlegende Features (wie zum Beispiel 
Kommandos, Interaktionen mit dem Editor, Einstellungen oder persistente
Datenspeicherung) sehr gut. Auch werden komplexe Spracherweiterungen
durch den Einsatz des \emph{Language Server Protocol} möglich.
Weiters bietet VS Code auch dezidierte Schnittstellen an, wenn Extensions
den Debugger oder das Testsystem von VS Code erweitern möchten.
Zur Anzeige bietet VS Code ein \emph{Tree View} und ein 
\emph{Webview}, in welchem beliebige Inhalte dargestellt werden können.

\subsection{IntelliJ IDEA}

Auch die IntelliJ-Plattform bietet viele grundlegende Funktionalitäten
für Plugins an (zum Beispiel Aktionen, Interaktion mit den geöffneten Dokumenten
und Projekten oder Einstellungen). Die Besonderheit, die IntelliJ
in Bezug auf Spracherweiterungen
mitbringt, ist das \emph{Program Structure Interface (PSI)}. 
Auf diese Weise können Sprachen gut in die IDEs integriert werden
und es kann extrem effizient mit dem Code interagiert werden.
Die Einbindung eines Debuggers oder Compilers sollte zwar grundsätzlich
möglich sein, allerdings gibt es hierzu (noch) keine Dokumentation.
Zur Anzeige wird auf Java Swing \emph{Tool Windows} gesetzt.
Weiters ist es möglich, in IntelliJ Plugins eigene 
\emph{Extension Points} zu deklarieren, die von weiteren Plugins
wiederum erweitert werden können.

\subsection{Vergleich}

Beide APIs sind sich in den angebotenen Grundlagen sehr ähnlich.
Der größte Unterschied liegt bei den Spracherweiterungen, da VS Code 
vollständig auf das Language Server Protocol setzt, während IntelliJ
dieses nicht verwendet. Für Plugin-EntwicklerInnen
kann es einen Unterschied machen, ob sie Benutzerschnittstellen
lieber mit Web-\linebreak 
Technologien wie HTML, CSS und JavaScript (Webviews) oder
mit Java Swing (Tool Windows) implementieren. 
Ein großer Vorteil den IntelliJ IDEA in Bezug auf den Feature-Umfang
bietet, ist das Erweiterungssystem mittels Extension Points.


\section{Einfachheit der API-Verwendung}
\label{sec:Vergleich_Intuitivität}

\subsection{Visual Studio Code}

Mit der VS Code Extension API wird ausschließlich
über das \emph{Extension Manifest} und das Modul \emph{vscode} interagiert.
Dieses Modul fasst alle Code-Schnittstellen zusammen und ist
in übersichtliche Abschnitte (z.B. commands, window, workspace) unterteilt.

\subsection{IntelliJ IDEA}

In IntelliJ werden Deklarationen im \emph{Plugin Configuration File} gemacht.
Die Schnittstellen, die vom Plugin implementiert werden müssen, und
die Klassen, mit denen interagiert wird, hängen von den deklarierten
\emph{Extension Points} ab. Es handelt sich dabei um unterschiedlichste
Klassen aus dem Quellcode der IntelliJ Platform.

\subsection{Vergleich}

Die VS Code API ist sehr übersichtlich und intuitiv und kann auch
schon nach kurzem Einlesen in die Dokumentation verwendet werden.
IntelliJ Plugins müssen mit vielen verschiedenen Klassen interagieren,
bei denen auf verschiedene Implementierungsdetails geachtet werden muss.
Dadurch kann auch die Dokumentation auf den ersten Blick komplex
wirken. Allerdings sind IntelliJ Plugins in ihren Möglichkeiten der
Interaktion mit der API weniger stark eingeschränkt.


\section{Dokumentation der API}
\label{sec:Vergleich_Dokumentation}

\subsection{Visual Studio Code}

Die Dokumentation für VS Code Extensions ist sehr übersichtlich.
In den ersten Abschnitten werden Grundlagen und häufig 
verwendete Features beschrieben.
In weiteren Abschnitten gibt es Anleitungen für spezielle Features,
Dokumentation für Spracherweiterungen und Beschreibungen
für das Testen und Publishen von Extension.
Weiters behandelt die Dokumentation auch eine vollständige Beschreibung
aller möglichen Interaktionen mit dem Extension Manifest
und eine vollständige Beschreibung der VS Code API Schnittstelle.
Auch ein Repository mit einfachen Code-Beispielen ist in der Dokumentation verlinkt.

\subsection{IntelliJ IDEA}

Die IntelliJ-Dokumentation beschreibt in den ersten Abschnitten kurz
die Grundlagen. Dabei werden vor allem im
Abschnitt \emph{Base Platform} viele häufig genutzte Funktionalitäten
für Plugins beschrieben. Die nächsten Abschnitte beschreiben
meist ein System der IntelliJ-Plattform und wie ein Plugin mit diesem
System interagieren kann. Zum Beispiel gibt es Abschnitte für \emph{Project Model}
oder \emph{PSI}. Leider findet man immer wieder Abschnitte mit der 
Bemerkung \enquote{Will be available soon}.
Es gibt zwar eine Liste von möglichen Extension Points, allerdings
finden sich in dieser keine Beschreibungen. Anstatt einer vollständigen
Dokumentation wird auf den IntelliJ-Plattform-Quellcode verwiesen.
Allerdings lassen auch die Kommentare im Code sehr zu wünschen übrig.
Auch für IntelliJ gibt es ein Repository mit Code-Beispielen, die sehr
hilfreich sein können.

\subsection{Vergleich}

VS Code bietet eine vollständige, übersichtliche und gut geschriebene 
Dokumentation für Plugin-EntwicklerInnen. Die IntelliJ-Dokumentation
bietet einen guten Überblick über die Systeme der IntelliJ-Plattform,
allerdings ist sie eher unübersichtlich und teilweise 
(an kritischen Stellen) unvollständig.
Beide Dokumentationen bieten Code-Beispiele an.

\section{Testbarkeit der Plugins}
\label{sec:Vergleich_Testbarkeit}

\subsection{Visual Studio Code}

VS Code erlaubt das Testen der Plugins und ermöglicht
dabei Zugriff auf die Schnittstellen der Plugin API.
Integrationstests werden mithilfe einer VS-Code-Instanz ausgeführt.

\subsection{IntelliJ IDEA}

IntelliJ ermöglicht das Testen des Plugins, in dem
von bestimmten Basis-Testklassen abgeleitet wird. Durch diese
Basisklassen werden zusätzliche Hilfsmethoden für die Tests
angeboten. Die Integrationstests werden in einer
headless-Instanz von IntelliJ IDEA ausgeführt, die großteils
einer echten Instanz von IntelliJ IDEA entspricht.

\subsection{Vergleich}

In IntelliJ müssen zwar einige Implementierungsdetails in den Testklassen
beachtet werden, dafür bietet IntelliJ eine bessere Unterstützung
durch Hilfsmethoden für Tests.


\section{Möglichkeiten des Publishings}
\label{sec:Vergleich_Publishing}

\subsection{Visual Studio Code}

Das Hochladen und Verwalten von Plugins im Marketplace ist
einfach und ausführlich dokumentiert. Das Werkzeug \emph{vsce}
bietet Möglichkeiten zum Verwalten von Extensions und 
erlaubt das automatisierte Publishing mittels CI/CD Pipelines.
Auf der Plugin-Seite des Marketplace kann eine \emph{Markdown}-Datei
angegeben werden, in der unter anderem auch Bilder oder Aufzählungen 
eingebunden und dargestellt werden können. 
Weiters bietet der Marketplace eine Seite für Reviews
und eine Q\&A-Seite für jedes Plugin.

\subsection{IntelliJ IDEA}

Da bei IntelliJ zu Beginn eine Signatur erstellt werden muss, ist das
erste (manuelle) Hochladen etwas aufwändiger. Allerdings
gibt es auch hier die Möglichkeit zum automatischen Publishing.
Die Plugin-Beschreibungen können in HTML formatiert werden. Es gibt
zusätzliche Felder um Bildschirmaufnahmen, ein Vorschauvideo oder
weitere Seiten in Markdown-Formatierung zur Marketplace-Seite
hinzuzufügen. Weiters gibt es die Möglichkeit kurze Beschreibungskarten
des Plugins erstellen zu lassen, die dann zum Beispiel auf 
einer eigenen Webseite eingebunden werden können.

\subsection{Vergleich}

Beide IDEs bieten einen ausgezeichneten Marketplace an
und ermöglichen das einfache Veröffentlichen eines Plugins.
Zur Darstellung des Plugins im Marketplace gibt es bei JetBrains
jedoch mehr Optionen.


\section{Installationsprozess der Plugins}
\label{sec:Vergleich_Installationsprozess}

\subsection{Visual Studio Code}

Die Installation des Plugins bei NutzerInnen funktioniert
über den Marketplace. Dieser ist in VS Code stark in den 
Editor eingebunden und es kann einfach nach 
Extensions gesucht werden. Das Installieren einer gewünschten
Extension funktioniert also über wenige Klicks.
Alternativ kann das Plugin auch auf der Marketplace Webseite
gesucht und von dort aus installiert werden.
Möchte man den VS Code Marketplace nicht nutzen, gibt es auch
die Möglichkeit, die Extension als \emph{.vsix}-Datei
zu verpacken und diese manuell von der Festplatte aus zu installieren.

\subsection{IntelliJ IDEA}

In IntelliJ IDEA kann ein Plugin über den JetBrains Marketplace
installiert werden. Der Marketplace ist dabei in den Einstellung
der IDE eingebunden. Auch hier kann das gewünschte Plugin einfach
gesucht und installiert werden.
Besucht man die Marketplace Webseite, so gibt es bei dieser nur einen
Download-Knopf, mit welchem das Plugin als \emph{.zip}-Datei heruntergeladen
werden kann. Diese gezippten Plugins können auch
unabhängig vom Marketplace erstellt und manuell
von der Festplatte aus installiert werden.

\subsection{Vergleich}

Beide Plattformen bieten zur Installation die selben Möglichkeiten.
Allerdings braucht man in VS Code einen Klick weniger, um eine
Extension zu finden, da der Marketplace direkt über ein Menü
und nicht nur über die Einstellungen erreichbar ist, wie in IntelliJ.


\section{Übersicht}

Die Tabelle \ref{tab:Ueberblick_Kriterien} zeigt 
eine Übersicht über die Bewertungskriterien und den
Vergleich der beiden Entwicklungsumgebungen. 
Ein Plus steht hierbei für die Entwicklungsumgebung,
die bei dem Kriterium mehr Vorteile bietet.

\begin{table}[hb]
    \caption{Bewertungsübersicht.}
    \label{tab:Ueberblick_Kriterien}
    \centering
    \small % Reduce font size
    \setlength{\tabcolsep}{10pt} % separator between columns (standard = 6pt)
    \renewcommand{\arraystretch}{1.25} % vertical stretch factor (standard = 1.0)

    \begin{tabular}{@{}llll@{}}
        \toprule
         & VS Code & IntelliJ Platform \\
        \midrule
        Popularität der Entwicklungsumgebung & + &  \\
        Ausführungszeit und Hardwareanforderungen & + &  \\
        Feature-Umfang &  & + \\
        Einfachheit der API-Verwendung & + &  \\
        Dokumentation der API & + &  \\
        Testbarkeit der Plugins &  & + \\
        Möglichkeiten des Publishings &  & + \\
        Installationsprozess der Plugins & + &  \\
        \bottomrule
    \end{tabular}
    
\end{table}
\chapter{Conclusion}
\label{cha:Conclusion}

Der direkte Vergleich der Systeme zeigt,
das beide APIs sehr unterschiedliche Vorzüge haben.

VS Code konnte vor allem durch die sehr intuitive
Schnittstelle überzeugen. Mir fiel die Implementierung
von neuen Features in der VS Code Extension meist sehr
leicht, wohingegen ich mich bei IntelliJ immer wieder
in die Dokumentation einlesen musste, um nicht
kleinigkeiten in der Implementierung zu übersehen.
Auch die allgemein hohe beliebtheit von VS Code
ermöglicht es das Plugin einer sehr breiten Masse 
an potenziellen NutzerInnen zur Verfügung zu stellen.
Insbesondere für EntwicklerInnen ohne Vorerfahrung
in der Plugin Entwicklung würde ich daher die Arbeit mit
Visual Studio Code empfehlen.

IntelliJ IDEA trumpft vor allem mit speziellen Features
wie der \emph{PSI}-Schnittstelle und den \emph{Extension Points},
die EntwicklerInnen viele Möglichkeiten bieten.
Auch die Popularität der JetBrains IDEs für spezifische 
Programmiersprachen kann ein ausschlaggebender Faktor
sein. Der größte Nachteil den IntelliJ für mich mitbringt
war die teils unvollständige und teils unverständliche
Dokumentation. Möchte man also ein Plugin Entwickeln, das 
für die Arbeit mit einer bestimmten Sprache (wie Java)
ausgerichtet ist, oder braucht das Plugin um nützlich zu
sein die Funktionen die ein vollständiges IDE bietet, so
sollte definitiv die IntelliJ Platform bevorzugt werden.

Insgesamt bieten beide IDEs eine ausgezeichnete 
Unterstützung für die Entwicklung von Plugins und
bringen viele grundlegende Features mit, die
ohne großem Aufwand durch Plugins wiederverwendet
werden können.

%%%-----------------------------------------------------------------------------
\appendix                                                               % Anhang 
%%%-----------------------------------------------------------------------------

% \chapter{Anhang A}
\label{app:AnhangA}

 % Technische Ergänzungen

%%%-----------------------------------------------------------------------------
\backmatter                          % Schlussteil (Quellenverzeichnis und dgl.)
%%%-----------------------------------------------------------------------------

\MakeBibliography % Quellenverzeichnis

%%%-----------------------------------------------------------------------------
% Messbox zur Druckkontrolle
%%%-----------------------------------------------------------------------------

%\include{back/messbox}

%%%-----------------------------------------------------------------------------
\end{document}
%%%-----------------------------------------------------------------------------
